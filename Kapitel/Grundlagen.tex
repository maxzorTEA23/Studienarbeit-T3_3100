%Kapitel Grundlagen 

\section{Grundlagen}
\label{cha: Grundlagen}
\subsection{Problemstellung}
Durch zahlreiche vorangegangene Studienarbeiten wurden bereits viele grundlegende Funktionen des Hochregallagers implementiert.\\
Hierzu gehören unter anderem:
\begin{itemize}
	\item automatisiertes Einlagern,
	\item manuelles Auslagern,
	\item sowie die \textbf{RFID-Materialerkennung}.
\end{itemize}
Diese Funktionen bleiben im Rahmen dieser Arbeit bestehen und sind nicht Hauptbestandteil der weiteren Bearbeitung.\\
Der Fokus der vorliegenden Studienarbeit liegt vielmehr auf der Integration zusätzlicher \textbf{Sicherheitskomponenten}, der Fertigstellung der \textbf{RFID-Materialerkennung} sowie der \textbf{Überarbeitung bzw. Umstellung des SPS-Programms} auf die Programmiersprache \textbf{Structured Text (ST)}.\\\\
Für den sicheren Betrieb eines Hochregallagers sind verschiedene Schutzmechanismen erforderlich, um Gefahren beim Eingriff in den laufenden Prozess zu vermeiden. Einige dieser Maßnahmen wurden bereits in früheren Arbeiten umgesetzt.\\
Ein wesentliches, bislang fehlendes Sicherheitselement ist jedoch eine Lichtschranke, die sich unmittelbar vor dem Lagerbereich befindet. Diese soll im Rahmen dieser Arbeit in das Steuerungsprogramm integriert werden, um potenzielle Gefährdungen bei manuellen Eingriffen während des Betriebs zu verhindern.\\\\
Darüber hinaus ist zur Gewährleistung eines fehlerfreien Ablaufs während der Materialerkennung die Funktionalität der RFID-Prüfung zu analysieren, zu optimieren und gegebenenfalls zu vervollständigen.\\
Ein weiterer wichtiger Aspekt betrifft die verwendete Programmiersprache des SPS-Systems. In den bisherigen Studienarbeiten wurde überwiegend die grafische Programmiersprache \textbf{FBS (Funktionsbausteinsprache)} eingesetzt. Diese bietet den Vorteil einer einfachen, blockorientierten Programmstruktur, stößt jedoch bei komplexeren Anwendungen schnell an ihre Grenzen.\\
Aus Gründen der Übersichtlichkeit, Wartbarkeit und Fehlersuche soll das bestehende Programm daher in die textbasierte Programmiersprache \textbf{ST (Structured Text)} übertragen werden.\\\\
Abschließend lässt sich festhalten, dass die wesentlichen Funktionen des bestehenden Hochregallagers bereits implementiert sind. 
Trotzdem bestehen noch Optimierungspotenziale im Hinblick auf Sicherheit, Struktur und Programmiermethodik. 
Um die geplanten Maßnahmen gezielt umsetzen zu können, ist es erforderlich, den aktuellen technischen Aufbau und die bestehende Programmstruktur zunächst detailliert zu analysieren. 
Im folgenden Abschnitt \enquote{Stand der Technik} werden daher die vorhandenen Komponenten und deren Funktion genauer betrachtet, um eine fundierte Grundlage für die anschließende Überarbeitung zu schaffen.
\subsection{Stand der Technik}



