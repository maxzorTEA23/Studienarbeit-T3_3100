\section{Implementierung eines Lichtgatters in das Gesamtkonzept}
Im aktuellen Ist-Zustand des Hochregallagers fehlt die vollständige Implementierung der Sicherheitslogik.\\
Lediglich der Not-Aus-Schalter ist hardwareseitig verdrahtet und schaltet bei Betätigung die Zuleitung ab. Es existiert jedoch keine automatische Abschaltung im Gefahrenfall.\\
Die Logik des Ein- und Auslagerns ist somit nicht gefahrenfrei. Sobald ein Eingriff in den Prozess erfolgt, kann dieser im aktuellen Zustand nicht automatisch gestoppt werden.\\
Da das Miniaturmodell eine reale Situation in der Logistik abbilden soll, stellt dieser Punkt ein erhebliches Defizit dar.\\
Es gibt verschiedene Möglichkeiten, um diesen Nachteil zu beheben. Die naheliegendste Option ist die Implementierung einer \textbf{Lichtgatter-Funktion}, um den Gefahrenbereich zuverlässig zu überwachen und den Betrieb im Störfall sicher zu unterbrechen.
Durch ein vorheriges Projekt wurde bereits, ein Modell der Firma IFM\cite{Lichtgatter_ifm}, installiert.
\subsection{Zielsetzung der Implementierung}
Die Studienarbeiten der vergangenen Jahre haben das Hochregallager in einen Zustand gebracht, der wichtige Funktionen wie Ein- und Auslagern abbildet.\\
Jedoch wurden hierbei, fälschlicherweise, die notwendigen Sicherheitsaspekte völlig außer Acht gelassen.\\
In einer realen Umgebung, beispielsweise in Materiallagern mit Hochregalsystemen großer Industrieunternehmen, wäre eine solche Vorgehensweise undenkbar.\\
Es muss jederzeit gewährleistet sein, dass Gefahren so gut wie möglich unter Kontrolle gehalten werden. Das bedeutet, im Fehlerfall muss jede gefährliche Bewegung sicher abgeschaltet werden.\\
Dies stellt die zentrale Zielsetzung der Implementierung dar. In jedem Betriebsmodus muss sichergestellt sein, dass bei Erkennung einer Gefahr automatisch abgeschaltet wird.\\
Dabei ist jedoch zu berücksichtigen, dass der Kolben des Zylinders, welcher die Materialien einlagert, kein Abschaltsignal auslösen darf. Dieser fährt nämlich unmittelbar in den Gatterbereich der Lichtschranke ein.\\
Somit ist zunächst eine umfassende Analyse des Ist-Zustands sowie der verschiedenen Prozesswerte des Lichtgatters erforderlich.\\
Erst im Anschluss kann die Implementierung der Sicherheitslogik erfolgen.

\subsection{Analyse der Parametrierung des Lichtgatters}
\label{cha:Lichtgatter}
Die Konfiguration des Lichtgatters ist entscheidend für die richtige Funktionsweise. Beim Modell von \textbf{IFM}\cite{Lichtgatter_ifm} gibt es mehrere Möglichkeiten zur Parametrierung.\\
Das Lichtgatter besitzt mehrere Lichtstrahlen, welche über einzelne Bits definiert sind. Sobald sich ein Objekt im Gatter befindet, werden mehrere dieser Lichtstrahlen gleichzeitig unterbrochen.\\
Der Sensor des vorliegenden Lichtgatters hat im sogenannten \textbf{Detektionsmodus (GTBO)} den Wert 1. Das bedeutet, dass er schaltet, sobald ein Objekt die Strahlen unterbricht.\\
Über \textbf{IO-Link} kann nun festgelegt werden, ab wie vielen unterbrochenen Strahlen der Sensor ein Schaltsignal ausgeben soll.
\subsubsection{Prozesswerte zur Steuerung des Arbeitsverhaltens}
Zur Steuerung des Verhaltens bei einer Unterbrechung stehen verschiedene Prozesswerte zur Verfügung\cite{Lichtgatter_ifm_2}:
\begin{itemize}
	\item \textbf{FBO (First Beam Occupied)}
	\item \textbf{LBO (Last Beam Occupied)}
	\item \textbf{CBO (Central Beam Occupied)}
	\item \textbf{NBO (Number of Beams Occupied)}
	\item \textbf{NCBO (Number of Consecutive Beams Occupied)}
\end{itemize}

Im Betriebsmodus \textbf{FBO} schaltet der Sensor bereits beim ersten unterbrochenen Lichtstrahl. Das bedeutet, Objekte, die im unteren Bereich des Lichtgatters liegen, werden als erstes erfasst.\\
Im Modus \textbf{LBO} geschieht genau das Gegenteil: Es wird beim letzten Lichtstrahl geschaltet.\\
Dieser Modus wird beispielsweise bei Endlagenerkennungen oder beim Auslagern verwendet – also immer dann, wenn das Austreten aus einem bestimmten Bereich überwacht werden soll.\\

Der Modus \textbf{CBO} ist etwas komplexer:
Hier wird der mittlere Lichtstrahl (bei mehreren zusammenhängenden Strahlen) überwacht. Tritt ein Objekt in das Lichtgatter ein, wird der zentrale Strahl als Referenzpunkt betrachtet. Bei mehreren Objekten wird immer vom größeren Objekt ausgegangen, also von dem, das mehr Strahlen gleichzeitig unterbricht – wie in Abbildung \ref{CBO_Modus} zu sehen.\\
Anwendungsbeispiele sind z. B. Durchgänge oder Förderabschnitte, also überall dort, wo Objekte mittig erfasst werden sollen (Referenzpunkt: Mitte).

\begin{figure}[H]
	\centering
	\includegraphics[width=0.7\linewidth]{images/CBO_Modus}
	\caption{Modus \textbf{CBO}; Quelle: \cite{Lichtgatter_ifm_2}}
	\label{CBO_Modus}
\end{figure}

Die letzten beiden Modi, \textbf{NBO} und \textbf{NCBO}, beschäftigen sich mit der konkreten Anzahl der Lichtstrahlen, die unterbrochen werden.\\
Der Unterschied der beiden Modi liegt darin, dass \textbf{NBO} die Gesamtanzahl der unterbrochenen Lichtstrahlen angibt, während \textbf{NCBO} die Anzahl der Strahlen beschreibt, die \textit{aufeinanderfolgend} unterbrochen werden.\\
Im Projekt wird der Modus \textbf{NBO} verwendet. Eine Abbildung soll die Funktionsweise noch einmal verdeutlichen (siehe Abbildung \ref{NBO_Modus}).
\begin{figure}[H]
	\centering
	\includegraphics[width=0.7\linewidth]{images/NBO_Modus}
	\caption{Funktionsweise des \textbf{NBO-Modus}; Quelle: \cite{Lichtgatter_ifm_2}}
	\label{NBO_Modus}
\end{figure}

\subsection{Komponentenübersicht und Systemaufbau}
Für die Umsetzung der Sicherheitsfunktion wurde ein Systemaufbau realisiert, der die Lichtschranke direkt mit einem IO-Link-Master und der SPS verbindet. Im Folgenden werden die eingesetzten Komponenten sowie deren Zusammenspiel innerhalb des Gesamtsystems beschrieben.\\[0.3cm]

\begin{table}[H]
	\centering
	\renewcommand{\arraystretch}{1.3}
	\begin{tabular}{|p{4cm}|p{4cm}|p{5cm}|p{2.5cm}|}
		\hline
		\textbf{Komponente} & \textbf{Hersteller / Typ} & \textbf{Funktion} & \textbf{Schnittstelle} \\ \hline
		Lichtgitter / Lichtschranke & IFM OY5100 (Infrarot) & Erfassung von Objekten und Überwachung des Gefahrenbereichs & IO-Link \\ \hline
		IO-Link Master & IFM AL1100 & Kommunikation zwischen IO-Link-Devices und der SPS & PROFINET / IO-Link \\ \hline
		SPS & Siemens S7-1500 & Steuerung des gesamten Systems, Verarbeitung der Sensorsignale & PROFINET \\ \hline
		Netzteil 24V DC & Siemens SITOP / vergleichbar & Versorgung von IO-Link-Master, Lichtschranke und SPS & 24V DC \\ \hline
		Aktorik (Zylinder / Motor) & Festo (pneumatisch) & Ausführung der Ein- und Auslagerbewegung & Digitale Ausgänge der SPS \\ \hline
	\end{tabular}
	\caption{Übersicht der verwendeten Komponenten im Systemaufbau}
	\label{tab:komponenten}
\end{table}

In Tabelle~\ref{tab:komponenten} sind die einzelnen Komponenten des Hochregallagers rund um die Lichtschranke dargestellt.\\
Diese müssen optimal miteinander interagieren, um die gewünschte Funktionalität sicherzustellen.\\
Die Lichtschranke ist vom Typ \textbf{OY5100} und stammt vom Hersteller \textbf{IFM}. Es handelt sich hierbei um ein steuerbares Lichtgitter, das über IO-Link mit der SPS kommuniziert.
\subsubsection{Systemaufbau}
Die folgende Abbildung zeigt den strukturellen Aufbau des Gesamtsystems in vereinfachter Form.\\[0.2cm]

\begin{figure}[H]
	%\hspace*{-1cm} % optional: Diagramm leicht nach links verschoben
	\begin{tikzpicture}[
		node distance=1.8cm,
		>=latex,
		font=\small,
		every node/.style={align=center}
		]
		% Nodes (vertikal angeordnet)
		\node (sps) [draw, rectangle, fill=orange!10, minimum width=3.5cm, minimum height=1cm] {SPS\\(Siemens S7-1500)};
		\node (master) [draw, rectangle, fill=green!10, below=of sps, minimum width=3.5cm, minimum height=1cm] {IO-Link Master\\(IFM AL1100)};
		\node (sensor) [draw, rectangle, fill=blue!10, below=of master, minimum width=3.5cm, minimum height=1cm] {Lichtgitter\\(IFM OY5100)};
		\node (actuator) [draw, rectangle, fill=red!10, below=of sensor, minimum width=3.5cm, minimum height=1cm] {Aktorik\\(Zylinder / Motor)};
		\node (power) [draw, rectangle, fill=gray!15, left=3cm of master, minimum width=3cm, minimum height=1cm] {Netzteil\\24V DC};
		
		% Signalfluss (von oben nach unten)
		\draw[->, thick] (sps) -- (master) node[midway, right]{\textbf{PROFINET}};
		\draw[->, thick] (master) -- (sensor) node[midway, right]{\textbf{IO-Link}};
		\draw[->, thick] (sensor) -- (actuator) node[midway, right]{\textbf{Signal / Reaktion}};
		
		% Spannungsversorgung (seitlich)
		\draw[->, thick] (power) -- (master) node[midway, above]{24V DC};
		\draw[->, thick] (power) |- (sensor);
		\draw[->, thick] (power) |- (sps);
		
	\end{tikzpicture}
	\caption{Vertikaler Systemaufbau des Lichtgatter-Systems}
	\label{fig:systemaufbau_vertikal}
\end{figure}

Im Diagramm~\ref{fig:systemaufbau_vertikal} ist der Aufbau der einzelnen Komponenten grafisch dargestellt. Der IO-Link-Master dient hierbei als Schnittstelle zwischen SPS und Lichtgitter und steuert die Kommunikation über das im TIA Portal implementierte Programm.\\
Im Fehlerfall werden gefährliche Bewegungen, die beispielsweise vom Zylinder oder den Motoren der Linearachsen ausgehen, automatisch gestoppt.\\
Im folgenden Abschnitt wird die Implementierung dieser Sicherheitslogik in der SPS detailliert erläutert.

\subsection{Implementierung der Sicherheitslogik in der SPS}
Die Sicherheitslogik wird, wie bereits erwähnt, in eine \textbf{Siemens S7-1500} SPS einprogrammiert. Die Programmierung erfolgt mithilfe der Software \textbf{TIA Portal}.\\
Für die Umsetzung der Arbeit wurde das bestehende Programm des Jahrgangs \textbf{TEA 22} als Grundlage verwendet. Zunächst wurde eine neue \textit{Program Organization Unit (POU)} mit der Bezeichnung \textbf{Lichtschranke} erstellt.\\
Im Anschluss erfolgte das Einbinden des Moduls \textbf{OY5100} im Reiter \textbf{Geräte und Netzwerke}. Der Adressbereich \textit{(frei wählbar)} wurde so konfiguriert, dass eine eindeutige Zuordnung innerhalb des Projekts gewährleistet ist.\\[0.2cm]

\begin{figure}[H]
	\centering
	\includegraphics[width=0.8\linewidth]{images/OY5100_Einbindung} % Beispielabbildung
	\caption{Einbindung der IFM-Lichtschranke OY5100 im TIA Portal}
	\label{fig:tia_device_network}
\end{figure}

Die zugewiesenen Adressen werden für die verschiedenen Prozesswerte genutzt (siehe Kapitel~\ref{cha: Prozesswerte}).\\
Als Vorbereitung auf die Programmierung wurde eine sogenannte \textit{Watchtable} erstellt. Diese dient dazu, die einzelnen Prozesswerte des Lichtgatters während der Laufzeit zu überwachen und deren Verhalten zu analysieren.\\
Die Prozesswerte liegen im Format \textbf{WORD} (2 Byte) vor und bilden die Anzahl der unterbrochenen Lichtstrahlen ab.\\[0.2cm]

Nach einer Reihe von Tests wurde festgestellt, dass der Kolben des Zylinders im Arbeitsbereich des Lichtgatters \textbf{drei Strahlen} gleichzeitig unterbricht. Hierfür erfolgte ein gezieltes Verfahren der Lineareinheit über alle vier Ebenen des Hochregallagers. So konnte überprüft werden, dass der Kolben in jeder Position konstant drei Strahlen abdeckt.\\[0.2cm]

Im weiteren Verlauf wurde diskutiert, welcher Prozesswert sich für die Sicherheitslogik am besten eignet. Es zeigte sich, dass der Modus \textbf{NBO (Number of Beams Occupied)} die Summe aller unterbrochenen Lichtstrahlen angibt (siehe Kapitel~\ref{cha: Prozesswerte}). Daher ist dieser Prozesswert optimal geeignet, um eine zuverlässige Abschaltung über das Lichtgatter zu realisieren.\\
Nach der Diskussion wurde, für die Realisierung der Abfrage, ein neuer FB (Freigabe\_Lichtschranke\_FB24) aufgebaut in \textbf{SCL}. Im Diagramm \ref{fig:fb24_einbindung} ist der Zusammenhang zwischen den einzelnen Modi und der \textbf{Freigabe} dargestellt.
\begin{figure}[H]
	\hspace*{3cm}
	\begin{tikzpicture}[
		node distance=4cm and 4cm,
		>=latex,
		font=\small,
		every node/.style={align=center}
		]
		
		% Zentrale Freigabe
		\node (fb24) [draw, circle, fill=blue!10, minimum width=2.8cm, align=center, thick] 
		{FB24\\Freigabe\\Lichtschranke};
		
		% Außenliegende FBS
		\node (fbhand) [draw, rectangle, fill=green!10, above=of fb24, minimum width=3.8cm, minimum height=1cm, thick]
		{FB\\Handbetrieb};
		\node (fbprozess) [draw, rectangle, fill=orange!10, right=of fb24, minimum width=4.3cm, minimum height=1cm, thick]
		{FB\\Prozesssteuerung\\(Ein-/Auslagerung)};
		\node (fbstoer) [draw, rectangle, fill=red!10, below=of fb24, minimum width=3.8cm, minimum height=1cm, thick]
		{FB\\Störungen};
		
		% Verbindungspfeile
		\draw[<->, thick] (fbhand.south) -- (fb24.north) node[midway, right]{Signalaustausch};
		\draw[<->, thick] (fbprozess.west) -- (fb24.east) node[midway, above]{Status / Freigabe};
		\draw[<->, thick] (fbstoer.north) -- (fb24.south) node[midway, right]{Rückmeldung / Fehler};
		
	\end{tikzpicture}
	\caption{Einbindung des Bausteins \texttt{FB24 – Freigabe\_Lichtschranke} in das Gesamtsystem}
	\label{fig:fb24_einbindung}
\end{figure}
Der Baustein \texttt{FB24 – Freigabe\_Lichtschranke} ist ein zentraler Bestandteil der Sicherheitsarchitektur.\\
Ohne ihn wäre ein sicheres Abschalten im Fehlerfall nicht möglich. Die \textbf{Prozesssteuerung} wird durch die Freigabe in jedem Schritt überwacht.\\
Durch die Einbindung in den \texttt{FB11 – Störungen} erfolgt eine Überwachung sämtlicher Ausgänge über die Variable \texttt{b\_Freigabe}. Diese wird gesetzt, sobald keine Störungen aktiv sind und die Freigabe durch das Lichtgitter gegeben ist.\\[0.2cm]

\begin{figure}[H]
	\centering
	\includegraphics[width=0.7\linewidth]{images/b_Freigabe}
	\caption{Setzen der Variable \texttt{b\_Freigabe} in \texttt{FB11 – Störungen}}
	\label{b_Freigabe}
\end{figure}
Jene Variable in Abbildung \ref{b_Freigabe} wurde bereits in vorherigen Projekten verwendet.\\
Sie wird im Block \texttt{FB\_Ausgänge} eingesetzt, als zusätzliche invertierte Rücksetzbedingung. Somit ist ein fehlerfreier Betriebsmodus der einzelnen Aktoren gewährleistet.
\subsubsection{Logische Abschaltbedingung der Lichtschranke}
\begin{figure}[H]
	\centering
	\begin{tikzpicture}[
		node distance=1.8cm,
		>=latex,
		font=\small,
		every node/.style={align=center},
		process/.style={draw, rectangle, rounded corners, fill=blue!10, minimum width=4cm, minimum height=1cm},
		decision/.style={draw, diamond, aspect=2, fill=orange!10, text width=3.8cm, align=center, inner sep=1pt},
		startstop/.style={draw, ellipse, fill=gray!15, minimum width=3cm, minimum height=1cm}
		]
		
		% Nodes
		\node (start) [startstop] {Start / Normalbetrieb};
		\node (read) [process, below=of start] {Prozesswert\\\texttt{NBO} auslesen};
		\node (decision) [decision, below=of read] {Ist \texttt{NBO > 3}?\\(3 = Zylinderhöhe)};
		\node (ok) [process, below left=1.5cm and 2.5cm of decision, fill=green!10] {Freigabe aktiv\\\texttt{b\_Freigabe = TRUE}};
		\node (stop) [process, below right=1.5cm and 2.5cm of decision, fill=red!10] {Aktorik abschalten\\\texttt{b\_Freigabe = FALSE}};
		\node (end) [startstop, below=of ok] {Zyklus fortsetzen};
		
		% Connections
		\draw[->, thick] (start) -- (read);
		\draw[->, thick] (read) -- (decision);
		\draw[->, thick] (decision) -- node[midway, left]{Nein} (ok);
		\draw[->, thick] (decision) -- node[midway, right]{Ja} (stop);
		\draw[->, thick] (ok) -- (end);
		
		
	\end{tikzpicture}
	\caption{Abschaltlogik der Lichtschranke basierend auf dem Prozesswert \texttt{NBO}}
	\label{fig:abschaltlogik_nbo}
\end{figure}
In Abbildung~\ref{fig:abschaltlogik_nbo} ist der Ablauf der Sicherheitsabschaltung grafisch dargestellt.
Sobald der Prozesswert \textbf{NBO} größer als 3 ist, wird die Aktorik abgeschaltet.
Das bedeutet, dass eine sofortige Abschaltung erfolgt, sobald zusätzlich zum Zylinder eine Hand oder ein anderes Objekt detektiert wird – also ein weiterer Lichtstrahl unterbrochen ist.