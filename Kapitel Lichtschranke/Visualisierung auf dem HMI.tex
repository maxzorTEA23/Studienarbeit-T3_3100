\subsection{Visualisierung auf dem HMI}
In diesem Kapitel wurde die Implementierung einer sicherheitsgerichteten Lichtschranke in das bestehende Hochregallager-System beschrieben. Dabei erfolgte zunächst die Analyse der Prozesswerte des IFM-Lichtgatters sowie die Auswahl des geeigneten Betriebsmodus zur sicheren Erkennung von Objekten im Gefahrenbereich.
Anschließend wurde der vollständige Systemaufbau erläutert und die Anbindung über den IO-Link-Master an die Siemens-SPS umgesetzt. Die daraus resultierende Sicherheitslogik ermöglicht es, gefährliche Bewegungen im Fehlerfall automatisch zu stoppen und so die funktionale Sicherheit des Systems zu gewährleisten.\\[0.2cm]

Die Visualisierung der Sicherheitszustände erfolgt über den integrierten Störungsbaustein des HMI-Panels, wodurch Unterbrechungen und Fehlermeldungen automatisch angezeigt werden. Eine Erweiterung dieser Funktion um eine gezielte Anzeige des Anlagenzustands (z.\,B. „Normalbetrieb“, „Sicherer Halt“) bietet zusätzliches Potenzial zur Verbesserung der Bedienerfreundlichkeit und Systemtransparenz.\\[0.2cm]

Damit ist die sicherheitstechnische Basis des Hochregallagers geschaffen, auf der im folgenden Kapitel weitere Funktionen und Erweiterungen aufgebaut werden können.

