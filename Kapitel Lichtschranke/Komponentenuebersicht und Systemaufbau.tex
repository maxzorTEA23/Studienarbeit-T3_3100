\subsection{Komponentenübersicht und Systemaufbau}
Für die Umsetzung der Sicherheitsfunktion wurde ein Systemaufbau realisiert, der die Lichtschranke direkt mit einem IO-Link-Master und der SPS verbindet. Im Folgenden werden die eingesetzten Komponenten sowie deren Zusammenspiel innerhalb des Gesamtsystems beschrieben.\\[0.3cm]

\begin{table}[H]
	\centering
	\renewcommand{\arraystretch}{1.3}
	\begin{tabular}{|p{4cm}|p{4cm}|p{5cm}|p{2.5cm}|}
		\hline
		\textbf{Komponente} & \textbf{Hersteller / Typ} & \textbf{Funktion} & \textbf{Schnittstelle} \\ \hline
		Lichtgitter / Lichtschranke & IFM OY5100 (Infrarot) & Erfassung von Objekten und Überwachung des Gefahrenbereichs & IO-Link \\ \hline
		IO-Link Master & IFM AL1100 & Kommunikation zwischen IO-Link-Devices und der SPS & PROFINET / IO-Link \\ \hline
		SPS & Siemens S7-1500 & Steuerung des gesamten Systems, Verarbeitung der Sensorsignale & PROFINET \\ \hline
		Netzteil 24V DC & Siemens SITOP / vergleichbar & Versorgung von IO-Link-Master, Lichtschranke und SPS & 24V DC \\ \hline
		Aktorik (Zylinder / Motor) & Festo (pneumatisch) & Ausführung der Ein- und Auslagerbewegung & Digitale Ausgänge der SPS \\ \hline
	\end{tabular}
	\caption{Übersicht der verwendeten Komponenten im Systemaufbau}
	\label{tab:komponenten}
\end{table}

In Tabelle~\ref{tab:komponenten} sind die einzelnen Komponenten des Hochregallagers rund um die Lichtschranke dargestellt.\\
Diese müssen optimal miteinander interagieren, um die gewünschte Funktionalität sicherzustellen.\\
Die Lichtschranke ist vom Typ \textbf{OY5100} und stammt vom Hersteller \textbf{IFM}. Es handelt sich hierbei um ein steuerbares Lichtgitter, das über IO-Link mit der SPS kommuniziert.
\subsubsection{Systemaufbau}
Die folgende Abbildung zeigt den strukturellen Aufbau des Gesamtsystems in vereinfachter Form.\\[0.2cm]

\begin{figure}[H]
	%\hspace*{-1cm} % optional: Diagramm leicht nach links verschoben
	\begin{tikzpicture}[
		node distance=1.8cm,
		>=latex,
		font=\small,
		every node/.style={align=center}
		]
		% Nodes (vertikal angeordnet)
		\node (sps) [draw, rectangle, fill=orange!10, minimum width=3.5cm, minimum height=1cm] {SPS\\(Siemens S7-1500)};
		\node (master) [draw, rectangle, fill=green!10, below=of sps, minimum width=3.5cm, minimum height=1cm] {IO-Link Master\\(IFM AL1100)};
		\node (sensor) [draw, rectangle, fill=blue!10, below=of master, minimum width=3.5cm, minimum height=1cm] {Lichtgitter\\(IFM OY5100)};
		\node (actuator) [draw, rectangle, fill=red!10, below=of sensor, minimum width=3.5cm, minimum height=1cm] {Aktorik\\(Zylinder / Motor)};
		\node (power) [draw, rectangle, fill=gray!15, left=3cm of master, minimum width=3cm, minimum height=1cm] {Netzteil\\24V DC};
		
		% Signalfluss (von oben nach unten)
		\draw[->, thick] (sps) -- (master) node[midway, right]{\textbf{PROFINET}};
		\draw[->, thick] (master) -- (sensor) node[midway, right]{\textbf{IO-Link}};
		\draw[->, thick] (sensor) -- (actuator) node[midway, right]{\textbf{Signal / Reaktion}};
		
		% Spannungsversorgung (seitlich)
		\draw[->, thick] (power) -- (master) node[midway, above]{24V DC};
		\draw[->, thick] (power) |- (sensor);
		\draw[->, thick] (power) |- (sps);
		
	\end{tikzpicture}
	\caption{Vertikaler Systemaufbau des Lichtgatter-Systems}
	\label{fig:systemaufbau_vertikal}
\end{figure}

Im Diagramm~\ref{fig:systemaufbau_vertikal} ist der Aufbau der einzelnen Komponenten grafisch dargestellt. Der IO-Link-Master dient hierbei als Schnittstelle zwischen SPS und Lichtgitter und steuert die Kommunikation über das im TIA Portal implementierte Programm.\\
Im Fehlerfall werden gefährliche Bewegungen, die beispielsweise vom Zylinder oder den Motoren der Linearachsen ausgehen, automatisch gestoppt.\\
Im folgenden Abschnitt wird die Implementierung dieser Sicherheitslogik in der SPS detailliert erläutert.
