\subsection{Zielsetzung der Implementierung}
Die Studienarbeiten der vergangenen Jahre haben das Hochregallager in einen Zustand gebracht, der wichtige Funktionen wie Ein- und Auslagern abbildet.\\
Jedoch wurden hierbei, fälschlicherweise, die notwendigen Sicherheitsaspekte völlig außer Acht gelassen.\\
In einer realen Umgebung, beispielsweise in Materiallagern mit Hochregalsystemen großer Industrieunternehmen, wäre eine solche Vorgehensweise undenkbar.\\
Es muss jederzeit gewährleistet sein, dass Gefahren so gut wie möglich unter Kontrolle gehalten werden. Das bedeutet, im Fehlerfall muss jede gefährliche Bewegung sicher abgeschaltet werden.\\
Dies stellt die zentrale Zielsetzung der Implementierung dar. In jedem Betriebsmodus muss sichergestellt sein, dass bei Erkennung einer Gefahr automatisch abgeschaltet wird.\\
Dabei ist jedoch zu berücksichtigen, dass der Kolben des Zylinders, welcher die Materialien einlagert, kein Abschaltsignal auslösen darf. Dieser fährt nämlich unmittelbar in den Gatterbereich der Lichtschranke ein.\\
Somit ist zunächst eine umfassende Analyse des Ist-Zustands sowie der verschiedenen Prozesswerte des Lichtgatters erforderlich.\\
Erst im Anschluss kann die Implementierung der Sicherheitslogik erfolgen.
