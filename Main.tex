\documentclass[a4paper,12pt]{scrartcl}
\usepackage{helvet}
\renewcommand{\familydefault}{\sfdefault}
\usepackage[a4paper,left = 2.5cm, right = 2.5cm,bottom=2.5cm, top=2.5cm]{geometry}
\usepackage[utf8]{inputenc}
\usepackage[T1]{fontenc}
\renewcommand{\thefigure}{\thesection.\arabic{figure}}
\setlength{\parindent}{0pt}
\usepackage{enumitem}
\usepackage{listings}
\usepackage{microtype} % hübschere Umbrüche
\lstset{
	basicstyle=\ttfamily\small,
	breaklines=true,
	breakatwhitespace=false,
	columns=fullflexible,
	keepspaces=true,
	upquote=true,
}
\newcommand{\code}[1]{\lstinline!#1!} % Kurzform für Inline-Code
\usepackage{csquotes}
\usepackage{graphicx}
\usepackage{subcaption}
\usepackage{wrapfig}
\usepackage{hyperref}
\usepackage{ifthen}
\usepackage{setspace}
\usepackage[backend=biber,style=numeric,sorting=nyt]{biblatex}
\DeclareNameAlias{default}{family-given}
\addbibresource{Literatur.bib}
\thispagestyle{plain}
\hypersetup{pageanchor=false}
\setcounter{tocdepth}{3}
\usepackage{float}
\usepackage{amsthm}
\usepackage{amsmath}
\newtheorem{definition}{Definition}[section]
\usepackage{ragged2e}
\usepackage{listings}
\usepackage{xcolor}
\lstset{
	basicstyle=\ttfamily\small,
	keywordstyle=\color{blue},
	commentstyle=\color{gray},
	stringstyle=\color{orange},
	numbers=left,
	numberstyle=\tiny,
	numbersep=5pt,
	frame=single,
	breaklines=true,
	captionpos=b
}
\usepackage[ngerman]{babel}
\usepackage[automark,headsepline,footsepline]{scrlayer-scrpage}
\clearpairofpagestyles
\automark[section]{section}
\ohead{}               % links leer
\ihead{\rightmark}  
\ofoot[\pagemark]{\pagemark}
\ifoot[Automatisiertes Modell- Hochregallager]{Automatisiertes Modell- Hochregallager}
\pagestyle{scrheadings}
\ModifyLayer[addvoffset=-1ex]{scrheadings.foot.above.line}
%\usepackage{chngcntr}
%\counterwithin{figure}{section}
%\counterwithin{table}{section}
%Kapitel + Abschnitt in Markierungen übernehmen
\begin{document}
	\pagenumbering{gobble}
	\begin{titlepage}
		\enlargethispage{4.0cm}
		\sffamily 								% Serifenlose Grundschrift für die Titelseite einstellen
		\parbox{0.5\linewidth}{
			\begin{flushleft}
				\includegraphics[width=0.4\linewidth]{images/DHBW_d_R_FN_46mm_4c.pdf}\\[5ex]
			\end{flushleft}
		}
		
		
		\begin{center}
			
			{\fontsize{20.74pt}{24pt}\selectfont
				\textbf{Automatisiertes Modell-Hochregallager}\\[1.5ex]}
			{\fontsize{14pt}{17pt}\selectfont
				\textbf{}\\[5ex]}
			{\fontsize{17pt}{20pt}\selectfont
				\textbf{T3\_3100}\\[2ex]}
			{\fontsize{14pt}{17pt}\selectfont
				Elektrotechnik\\[2ex]}
			{\fontsize{12pt}{14pt}\selectfont
				Automation\\[1ex]
				Duale Hochschule Baden-Württemberg Ravensburg, Campus Friedrichshafen\\[5ex]
				von\\[1ex]
				Maximilian Zorko und Willi Schaal\\[15ex]}
			
			
		\end{center}
		
		\begin{flushleft}
			{\fontsize{12pt}{14pt}\selectfont
				\begin{tabular}{ll}
					Abgabedatum:					& \quad  \\
					Bearbeitungszeitraum:		   	& \quad 16.10.2025 -    \\ 
					Matrikelnummer: 			& \quad 3960407/5732737 \\ 
					Kurs: 							& \quad TEA 23 \\
					% entfällt bei Studienarbeit
					Betreuerin / Betreuer:  & \quad Thorsten Kever \\ % Betreuerin / Betreuer der Arbeit
					%	Gutachterin / Gutachter: & \quad \gutachterdhbw \\ [2ex] % Gutachterin / Gutachter der DHBW (nur bei der Bachelorarbeit erforderlich)
				\end{tabular}
			}
		\end{flushleft}
		%%%%% Nachfolgende Zeilen einkommentieren, wenn Copyrightvermerk gewünscht ist
		\begin{flushleft}
			{\fontsize{11pt}{13pt}\selectfont
				Copyrightvermerk:\\
				Dieses Werk einschließlich seiner Teile ist \textbf{urheberrechtlich geschützt}. Jede Verwertung außerhalb der engen Grenzen des Urheberrechtgesetzes ist ohne Zustimmung des Autors unzulässig und strafbar. Das gilt insbesondere für Vervielfältigungen, Übersetzungen, Mikroverfilmungen sowie die Einspeicherung und Verarbeitung in elektronischen Systemen.
			}
		\end{flushleft}
		\begin{flushright}
			{\fontsize{11pt}{13pt}\selectfont \copyright{} 2025 }
		\end{flushright}
	\end{titlepage}
	
\clearpage
\markright{Inhaltsverzeichnis}
\tableofcontents
\newpage
\cleardoublepage
\pagenumbering{arabic}	
\setcounter{page}{1}
% Kapitel Einleitung

\section{Einleitung}
\label{cha: Einleitung}
\subsection{Motivation}
In Zeiten von \textbf{Industrie 4.0} hat die Bedeutung des Automatisierungsgrades in den vergangenen Jahren erheblich zugenommen.\\
Besonders im Hinblick auf Effizienz, Sicherheit und die Effektivität automatisierter Abläufe spielt die Automatisierung eine zentrale Rolle.\\
Der Begriff \textbf{Industrie 4.0} beschreibt den aktuellen Wandel in der industriellen Produktion hin zu intelligent vernetzten Systemen. Dabei spielt die Automatisierung eine entscheidende Rolle. Die Kommunikation der einzelnen Maschinen, Produktionsanlagen und Lager erfolgt hierbei über \textbf{industrielle Netzwerke}, was zu einem echtzeitfähigen Verhalten beitragen kann.\\
Durch die zunehmende Digitalisierung und Vernetzung industrieller Prozesse entstehen stetig neue Anforderungen an die Steuerungs- und Regelungstechnik. Diese Entwicklungen führen zu immer komplexeren, aber auch effizienteren Produktions- und Logistiksystemen.\\\\
Gerade im Bereich der Lagerlogistik ist es entscheidend, sämtliche Sicherheitsmaßnahmen einzuhalten, um einen reibungslosen und sicheren Ablauf zu gewährleisten. Gleichzeitig wird eine präzise und fehlerarme Handhabung von Materialien ermöglicht, was wesentlich zur Prozesssicherheit und Produktqualität beiträgt. Lagerprozesse sind ein wesentlicher Bestandteil des Materialflusses innerhalb der Produktion. Fehler in diesem Bereich können zu Stillständen und hohen Kosten führen. Um solche Probleme zu vermeiden, ist ein reibungsloser Ablauf aller Prozessschritte erforderlich.
\subsection{Bedeutung der Lagerlogistik}
Im Bereich der Automatisierungstechnik existieren zahlreiche Konzepte, um Produkte und Ersatzteile effizient zu lagern und zu verwalten.\\
Eine klassische Variante stellt die manuelle Lagerung in großen Lagerräumen dar. Dabei sind jedoch umfangreiche Such- und Zuordnungsprozesse erforderlich, um die gewünschten Komponenten zu identifizieren und zu entnehmen.\\
Eine wesentlich effektivere Lösung bietet die automatisierte Lagertechnik, insbesondere das sogenannte \enquote{Hochregallager}.\\
Ein Hochregallager ermöglicht die automatisierte Ein- und Auslagerung von Materialien und Ersatzteilen. Solche Systeme ermöglichen es durch den Einsatz moderner Sensorik, wie etwa der Materialerkennung über \textbf{RFID} oder der automatischen Auftragsausführung mittels QR-Code-Erkennung, Materialien präzise zu lagern und zu verwalten. Des Weiteren können die Prozesse über geeignete Softwarelösungen überwacht und gesteuert werden.\\
Dadurch können Prozesszeiten verkürzt, Fehlerquoten reduziert und die Übersichtlichkeit innerhalb des Lagers deutlich verbessert werden.\\\\
Grundsätzlich lässt sich die Handhabung in zwei Varianten unterteilen:
\begin{itemize}
	\item Automatische Ein- und Auslagerung
	\item Manuelle Ein- und Auslagerung
\end{itemize}
Das Hochregallager ist in der Regel in einem separaten Bereich installiert, in dem autorisierte Mitarbeiter mithilfe eines Auftrags- oder Identifikationssystems auf die eingelagerten Komponenten zugreifen können.\\
Die Automatisierung solcher Systeme ermöglicht eine zuverlässige, reproduzierbare und sichere Abwicklung logistischer Prozesse. Hierdurch werden menschliche Fehler weitestgehend reduziert und somit die Sicherheit erhöht.
\subsection{Das DHBW-Hochregallager-Modell}
Zur Simulation und Demonstration der grundlegenden Lagerfunktionen wurde an der Dualen Hochschule Baden-Württemberg (DHBW) eine Miniaturversion eines Hochregallagers entwickelt, die die wesentlichen Prozesse der Ein- und Auslagerung abbildet.\\
Das Modell eignet sich hervorragend als Lern- und Übungsplattform für Studierende, um praxisnah Kenntnisse in der industriellen Automatisierung zu erwerben. Ziel des Systems ist es, die Zusammenarbeit der einzelnen Sensoren und Aktoren zu simulieren und dadurch den Studierenden einen praxisnahen Einblick in die industrielle Automatisierung zu bieten.\\\\
Das System wurde in den vergangenen Jahren bereits mehrfach für Studien- und Projektarbeiten im Studiengang \enquote{Elektrotechnik – Automation} eingesetzt und kontinuierlich weiterentwickelt. Diese Arbeiten ermöglichten es den Studierenden, ein vertieftes Verständnis für Steuerungs- und Automatisierungssysteme zu erlangen und praxisorientierte Erfahrungen zu sammeln.
\clearpage
\subsection{Zielsetzung der Studienarbeit}
Trotz der bisherigen Entwicklungen besteht weiterhin Verbesserungspotenzial, insbesondere im Bereich der Sicherheit und der Programmstruktur.\\
Die vorliegende Studienarbeit befasst sich daher mit der Überarbeitung und Einführung neuer Sicherheitsmaßnahmen. Darüber hinaus soll eine auftragsbasierte Materialein- und -auslagerung implementiert werden. Weitere Schwerpunkte bilden die vollständige Implementierung der \textbf{RFID}-basierten Materialerkennung sowie die Umstellung des bestehenden SPS-Programms auf die Programmiersprache \textbf{Structured Text (ST)}.\\\\
Aufgrund des hohen Umfangs und der technischen Komplexität des Projekts erfolgt die vollständige Umsetzung in den \textbf{Theoriephasen 5 und 6}.\\
Ziel der aktuellen Bearbeitung (Theoriephase 5) ist die Einführung neuer Sicherheitskonzepte und eines neuen Umlagerung- Prozesses.\\\\
Abschließend soll die durchgeführte Arbeit die Grundlage für die vollständige Realisierung eines funktionsfähigen, sicheren und didaktisch wertvollen Hochregallagermodells bilden, das zukünftigen Studierenden als praxisnahes Lehr- und Forschungsobjekt dient.\\
Die in den folgenden Kapiteln erläuterten Schritte zur Planung und Umsetzung sollen verdeutlichen, wie die Studienarbeit (Teil 1) zur Weiterentwicklung der Funktionalität und der Sicherheit des Systems beiträgt. Nun erfolgt eine Erläuterung der Problemstellung und des aktuellen Standes der Technik.
%Kapitel Grundlagen 

\section{Grundlagen}
\label{cha: Grundlagen}
\subsection{Problemstellung}
Durch zahlreiche vorangegangene Studienarbeiten wurden bereits viele grundlegende Funktionen des Hochregallagers implementiert.\\
Hierzu gehören unter anderem:
\begin{itemize}
	\item automatisiertes Einlagern,
	\item manuelles Auslagern,
	\item sowie die \textbf{RFID-Materialerkennung}.
\end{itemize}
Diese Funktionen bleiben im Rahmen dieser Arbeit bestehen und sind nicht Hauptbestandteil der weiteren Bearbeitung.\\
Der Fokus der vorliegenden Studienarbeit liegt vielmehr auf der Integration zusätzlicher \textbf{Sicherheitskomponenten}, der Fertigstellung der \textbf{RFID-Materialerkennung} sowie der \textbf{Überarbeitung bzw. Umstellung des SPS-Programms} auf die Programmiersprache \textbf{Structured Text (ST)}.\\\\
Für den sicheren Betrieb eines Hochregallagers sind verschiedene Schutzmechanismen erforderlich, um Gefahren beim Eingriff in den laufenden Prozess zu vermeiden. Einige dieser Maßnahmen wurden bereits in früheren Arbeiten umgesetzt.\\
Ein wesentliches, bislang fehlendes Sicherheitselement ist jedoch eine Lichtschranke, die sich unmittelbar vor dem Lagerbereich befindet. Diese soll im Rahmen dieser Arbeit in das Steuerungsprogramm integriert werden, um potenzielle Gefährdungen bei manuellen Eingriffen während des Betriebs zu verhindern.\\\\
Darüber hinaus ist zur Gewährleistung eines fehlerfreien Ablaufs während der Materialerkennung die Funktionalität der RFID-Prüfung zu analysieren, zu optimieren und gegebenenfalls zu vervollständigen.\\
Ein weiterer wichtiger Aspekt betrifft die verwendete Programmiersprache des SPS-Systems. In den bisherigen Studienarbeiten wurde überwiegend die grafische Programmiersprache \textbf{FBS (Funktionsbausteinsprache)} eingesetzt. Diese bietet den Vorteil einer einfachen, blockorientierten Programmstruktur, stößt jedoch bei komplexeren Anwendungen schnell an ihre Grenzen.\\
Aus Gründen der Übersichtlichkeit, Wartbarkeit und Fehlersuche soll das bestehende Programm daher in die textbasierte Programmiersprache \textbf{ST (Structured Text)} übertragen werden.\\\\
Abschließend lässt sich festhalten, dass die wesentlichen Funktionen des bestehenden Hochregallagers bereits implementiert sind. 
Trotzdem bestehen noch Optimierungspotenziale im Hinblick auf Sicherheit, Struktur und Programmiermethodik. 
Um die geplanten Maßnahmen gezielt umsetzen zu können, ist es erforderlich, den aktuellen technischen Aufbau und die bestehende Programmstruktur zunächst detailliert zu analysieren. 
Im folgenden Abschnitt \enquote{Stand der Technik} werden daher die vorhandenen Komponenten und deren Funktion genauer betrachtet, um eine fundierte Grundlage für die anschließende Überarbeitung zu schaffen.
\subsection{Stand der Technik}
\label{cha:Stand der Technik}
\subsubsection{Aufbau des Hochregallagers}

Bei der zu optimierenden Anlage handelt es sich um ein \textbf{Modell-Hochregallager} im Laborraum H001 der \textbf{Dualen Hochschule Baden-Württemberg} am Campus Friedrichshafen.  
Diese dient dazu, den Studierenden bei vertretbaren Anschaffungskosten ein Modell einer \textbf{modernen Hochregalanlage}, wie sie auch in zahlreichen \textbf{Industriebetrieben} zu finden ist, zu bieten.

Um den Aufbau der Anlage fachgerecht erklären zu können, ist die Definition eines \textbf{Koordinatensystems} für die Bewegung der \textbf{Einlagerungsvorrichtung} notwendig:

\tdplotsetmaincoords{70}{120}  % Perspektive: Elevation und Azimut

% Perspektive: Elevation = 70°, Azimut = 0° (x nach rechts, y nach hinten, z nach oben)
\tdplotsetmaincoords{70}{0}
\begin{center} % Zentriert die Grafik
\begin{tikzpicture}[tdplot_main_coords, scale=2]
	
	
	% Achsen
	\draw[->, thick] (0,0,0) -- (3,0,0) node[anchor=north east]{$x$ Druckluftzylinder};
	\draw[->, thick] (0,0,0) -- (2,3,0) node[anchor=north west]{$y$ Linearelement};
	\draw[->, thick] (0,0,0) -- (0,0,3) node[anchor=south]{$z$ Kugelumlaufspindel};
	
	  % Quadrat auf der yz-Ebene (x=0, zwischen y und z)
	 % Punkte: Ursprung, y-Achse, y+z, z-Achse
	 \fill[blue!10, opacity=0.7] 
	 (0,0,0) -- (2,3,0) -- (2,3,3) -- (0,0,3) -- cycle;
	 
	 % Beschriftung der Fläche
	 \node at (1,1.5,1.5) {\textbf{Regalfassade}};
	


\end{tikzpicture}
\end{center}

\begin{itemize}
	\item x-Achse -> Bewegung des Trägers in das Hochregal
	\item y-Achse -> Bewegung des Trägers in Horizontale entlang der Regalstruktur
	\item z-Achse -> Bewegung des Trägers in Vertikale entlang der Regalstruktur
\end{itemize}

Die \textbf{Förder- und Bewegungssysteme} der Modell-Hochregalanlage stellen das zentrale Element zur Realisierung \textbf{automatisierter Lagerprozesse} dar. Die Anlage ist so konzipiert, dass sie die \textbf{Ein- und Auslagerung} von Lagergütern in einem mehrstöckigen Regalsystem ermöglicht. Die Bewegungsachsen werden in drei Dimensionen unterteilt: \textbf{horizontal}, \textbf{vertikal} und die eigentliche \textbf{Einlagerbewegung}.
\\

Die \textbf{horizontale Verfahrbewegung} (y-Achse) des Regalbediengeräts erfolgt über ein \textbf{Linearelement}, das entlang einer \textbf{Führungsschiene} verläuft. Das Linearsystem basiert auf einem \textbf{motorgetriebenen Schlitten}, welcher präzise Positionierungen entlang der Regalfassade ermöglicht. Die Führungsschiene gewährleistet eine stabile und reibungsarme Bewegung, während der Antrieb über einen \textbf{Schrittmotor} erfolgt, der über eine \textbf{Steuerungseinheit} angesteuert wird. Die Positionserfassung wird gegenwärtig durch \textbf{Timer im Steuerungsprogramm} realisiert. Die Positionierung der \textbf{Endschalter} erfolgt ausschließlich am Ende der Achsen.
\\

Die \textbf{vertikale Bewegung} (z-Achse) wird durch eine \textbf{Kugelumlaufspindel} realisiert, die eine hohe Positioniergenauigkeit und mechanische Effizienz bietet. Die Spindel ist mit einem \textbf{rotatorischen Antrieb} gekoppelt, der die Drehbewegung in eine \textbf{lineare Hubbewegung} umsetzt. Die Verwendung von \textbf{Kugelumlaufmuttern} führt zu einer Minimierung der Reibung und einer Erhöhung der Tragfähigkeit, was insbesondere bei mehrstöckigen Regalsystemen vorteilhaft ist. Die vertikale Führung erfolgt über ein stabiles Profil, das die Bewegung des Hubsystems stabilisiert und ein Verkanten verhindert.
\\

Die eigentliche \textbf{Einlagerung des Lagerguts} (x-Achse) erfolgt mittels eines \textbf{Druckluftzylinders}. Der Zylinder ist am Regalbediengerät montiert und fährt bei Erreichen der Zielposition aus, um das Lagergut in das vorgesehene Fach zu überführen. Die Wahl eines pneumatischen Systems ermöglicht eine schnelle und kraftvolle Bewegung, die unabhängig von der elektrischen Steuerung arbeitet und sich gut für wiederholte, gleichförmige Bewegungsabläufe eignet. Die Steuerung des Zylinders erfolgt durch ein \textbf{Magnetventil}, das durch die SPS angesteuert wird.

\textbf{INSERT PICTURE}

Die \textbf{Sensorik und Steuerungseinheiten} der Modell-Hochregalanlage übernehmen zentrale Aufgaben zur Gewährleistung eines sicheren und präzisen Betriebs. Sie ermöglichen sowohl die \textbf{Positionsbestimmung} der beweglichen Komponenten als auch die \textbf{Identifikation der Lagergüter} und die \textbf{Prozesssteuerung}.
\\

Zur Erfassung der Endstellungen der horizontalen und vertikalen Bewegungsachsen werden mechanische \textbf{Endschalter} verwendet. Diese sind jeweils an den Begrenzungspunkten der Lineareinheit sowie der Kugelumlaufspindel angebracht und dienen der sicheren Detektion der maximalen Verfahrwege. Das Erreichen einer Endlage löst den entsprechenden Schalter aus und führt zur Übertragung eines Signals an die Steuerungseinheit. Diese Vorgehensweise verhindert ein Überfahren der mechanischen Grenzen und gewährleistet den Schutz der Anlage vor potenziellen Schäden.
\\

Die \textbf{Materialerfassung} erfolgt mittels eines \textbf{RFID-Systems}, welches sich aus einer \textbf{RFID-Antenne} und entsprechenden \textbf{Transpondern} zusammensetzt. Jedes Element des Lagers ist mit einem \textbf{RFID-Tag} ausgestattet, der eine eindeutige Identifikation ermöglicht. Die Positionierung der Antenne erlaubt es, beim Ein- oder Auslagern eines Artikels dessen Transponder auszulesen. Die erfassten Daten werden unmittelbar an die Steuerung übermittelt und dort verarbeitet. Das berührungslose Identifikationsverfahren ermöglicht eine schnelle und fehlerfreie Zuordnung der Lagergüter.
\\

Die zentrale Steuerung der Anlage erfolgt mittels einer \textbf{SPS vom Typ Siemens S7-1511}, welche sämtliche Bewegungsabläufe, Sensorrückmeldungen und Aktorsteuerungen koordiniert. Die SPS ist über ein \textbf{HMI (Human-Machine Interface)} mit dem Bedienpersonal verbunden, wodurch eine intuitive Bedienung und Visualisierung der Anlagenzustände ermöglicht wird. Über das HMI können Betriebsmodi gewählt, Lagerprozesse gestartet und Diagnosedaten eingesehen werden.

\input{Kapitel Grundlagen/Grundlagen-SIemensTIA.tex}
\subsubsection{SPS-Programmierung}
In der Automatisierungstechnik werden speicherprogrammierbare Steuerungen (SPS) zur Realisierung unterschiedlichster Abläufe eingesetzt.  
Um die gewünschten Funktionen zur Automatisierung von Prozessen umzusetzen, ist es erforderlich, diese Abläufe in Form eines Programms in die Steuerung zu implementieren.

Ein SPS-Programm folgt dabei in der Regel einem klar strukturierten Aufbau \cite{Seitz.2021}:
\begin{itemize}
	\item Definition der Ein-, Ausgangs- und Hilfsvariablen in einer Variablentabelle,
	\item Programmierung des Hauptprogramms mit den zugehörigen Unterprogrammen und Funktionsbausteinen,
	\item Kompilierung des Programms und Übertragung in die Steuerung.
\end{itemize}

\begin{figure}[H]
	\centering
	\includegraphics[width=0.8\linewidth]{images/Variablentabelle.jpg}
	\caption{Übersicht der Variablentabelle in TIA Portal \cite{siemens_vars2024}}
	\label{fig:uebersicht_tia}
\end{figure}

Das Hauptprogramm besteht aus sogenannten \textbf{POUs} (Program Organization Units, deutsch: Programmorganisationseinheiten).  
Diese werden in drei Arten unterteilt: \textbf{Programme}, \textbf{Funktionen (FC)} und \textbf{Funktionsbausteine (FB)}.  
Die Realisierung dieser POUs in \textit{Siemens TIA Portal} wurde bereits im vorherigen Kapitel (vgl. \ref{label}) beschrieben.  
Nachfolgend wird die Bedeutung und Funktion der einzelnen Organisationseinheiten erläutert.

Das \textbf{Programm} dient der Realisierung komplexer Abläufe, beispielsweise von Schrittketten oder Hauptlogiken.  
Eine \textbf{Funktion (FC)} ist mit einer C-ähnlichen Funktion vergleichbar. Sie verarbeitet eine oder mehrere Eingangsvariablen und gibt nach der Abarbeitung ein Ergebnis zurück.  
Gemäß IEC 61131-3 können Funktionen jedoch nicht rekursiv aufgerufen werden und besitzen kein eigenes Speicherverhalten, d.\,h. sie können keine Werte dauerhaft speichern \cite{Seitz.2021}.  
Ein einfaches Beispiel hierfür ist ein logisches \textbf{UND-Gatter}.

Ein \textbf{Funktionsbaustein (FB)} hingegen besitzt ein internes Gedächtnis und kann Werte über mehrere Zyklen hinweg speichern. Dadurch eignet er sich besonders für Zustandsautomaten oder speichernde Operationen.  
Ein klassisches Beispiel hierfür ist ein \textbf{RS-Flip-Flop}.

Für die Programmierung stehen verschiedene Sprachen nach IEC 61131-3 \cite{IEC6113-3_2022} zur Verfügung:
\begin{itemize}
	\item KOP (Kontaktplan),
	\item AWL (Anweisungsliste),
	\item FUP/FBS (Funktionsplan / Funktionsbausteinsprache),
	\item ST/SCL (Structured Text / Structured Control Language)\footnote{SCL ist die Siemens-spezifische Implementierung der nach IEC~61131-3 genormten Programmiersprache Structured Text (ST) im TIA Portal \cite{Kanngießer}.}
\end{itemize}

In modernen Industrieanlagen werden überwiegend \textbf{FUP} und \textbf{ST} eingesetzt. Diese Programmiersprachen zeichnen sich durch eine gute Übersichtlichkeit und eine effiziente Fehleranalyse aus.  
Trotzdem bestehen zwischen beiden Varianten wesentliche Unterschiede, die in Tabelle~\ref{tab:vergleich_fup_st} dargestellt sind.

\begin{table}[H]
	\centering
	\begin{tabular}{|p{7cm}|p{7cm}|}
		\hline
		\textbf{ST / SCL (Structured Text / Structured Control Language)} & \textbf{FUP (Funktionsplan)}\\
		\hline
		Textbasierte Programmiersprache & Grafische Programmiersprache\\
		\hline
		Hochsprachenähnlich (z.\,B. C, Pascal) & Verwendung grafischer Symbole und logischer Verknüpfungen\\
		\hline
		Klare und kompakte Struktur & Gefahr der Unübersichtlichkeit bei komplexen Programmen\\
		\hline
		Besonders geeignet für mathematische und logische Operationen & Gut geeignet für einfache logische Abläufe\\
		\hline
	\end{tabular}
	\caption{Vergleich zwischen ST (Structured Text) und FUP (Funktionsplan) \cite{siemens_programmieren}}
	\label{tab:vergleich_fup_st}
\end{table}

Wie aus der Tabelle hervorgeht, bietet die textbasierte Programmiersprache \textbf{ST} insbesondere bei komplexen Strukturen und mathematischen Berechnungen erhebliche Vorteile.  
\textbf{FUP} hingegen ist für Einsteiger leichter verständlich und eignet sich für überschaubare Steuerungslogiken.  
Mit zunehmender Komplexität der Abläufe stößt FUP jedoch schnell an seine Grenzen, da die grafische Darstellung umfangreicher Strukturen unübersichtlich werden kann \cite{siemens_programmieren}.  
Um die genauen Unterschiede zu erläutern, sollen zunächst grundlegende Elemente der einzelnen Sprachen eingeführt werden.  
Zunächst erfolgt eine Beschreibung der wichtigsten Grundbausteine in FUP.

\subsubsection*{Wichtige Grundfunktionen (UND, ODER, NICHT) in FUP}
\begin{figure}[H]
	\centering
	\begin{subfigure}[b]{0.49\textwidth}
		\centering
		\includegraphics[width=0.8\linewidth]{images/UND_GATTER}
		\caption{\textbf{UND\_Gatter} \cite{Hering}}
	\end{subfigure}
	\hfill
	\begin{subfigure}[b]{0.49\textwidth}
		\centering
		\includegraphics[width=0.8\linewidth]{images/ODER_GATTER}
		\caption{\textbf{ODER\_Gatter} \cite{Hering}}
	\end{subfigure}
	
	\vspace{1cm}
	
	\begin{subfigure}[b]{0.49\textwidth}
		\centering
		\includegraphics[width=0.8\linewidth]{images/NOT_GATTER}
		\caption{\textbf{NOT\_Gatter} \cite{Hering}}
	\end{subfigure}
	\caption{Darstellung der logischen Grundfunktionen in FUP}
	\label{fig:grundfunktionen_fup}
\end{figure}

\subsubsection*{Wichtige Funktionsbausteine in FUP}
\begin{figure}[H]
	\centering
	\begin{subfigure}[b]{0.49\textwidth}
		\centering
		\includegraphics[width=0.8\linewidth]{images/SR_FLIPFLOP}
		\caption{\textbf{SR\_FLIPFLOP} \cite{Hering}}
	\end{subfigure}
	\hfill
	\begin{subfigure}[b]{0.49\textwidth}
		\centering
		\includegraphics[width=0.8\linewidth]{images/TON}
		\caption{\textbf{Einschaltverzögerung (TON)} \cite{Hering}}
	\end{subfigure}
	\caption{Darstellung wichtiger Funktionsbausteine in FUP}
	\label{fig:funktionsbausteine_fup}
\end{figure}

In den obigen Abbildungen sind die wichtigsten Grundfunktionen und Funktionsbausteine in \textbf{FUP} dargestellt.  
Mit diesen können diverse Abläufe realisiert werden, wie beispielsweise einfache Förderbandsteuerungen.  
Komplexe, lange Abläufe sind jedoch sehr aufwendig zu realisieren. Für eine SPS ist es daher gebräuchlich, hierfür auf \textbf{ST/SCL} auszuweichen.  
In FUP können ebenfalls mathematische Operationen durchgeführt werden. Für Berechnungen existieren spezielle Bausteine. Allerdings ist es schwierig, längere Rechenvorgänge übersichtlich darzustellen, da die Struktur schnell unübersichtlich wird.  
Aus diesem Grund werden für sämtliche mathematische Operationen häufig Berechnungen in \textbf{ST/SCL} durchgeführt.  
Beim vorliegenden Modell, dem Hochregallager, wurde beispielsweise die Positionsermittlung der Lagerplätze für das automatisierte Einlagern im vergangenen Studienprojekt in \textbf{SCL} realisiert.
\subsubsection*{Grundlegende Anweisungen in SCL}
Im vorliegenden Projekt wird mit einer Siemens SPS gearbeitet. Somit ist SCL (Structured Control Language) der Standard im Bereich textbasierte Programmierung.\\
Hierbei gibt es einige konkrete Anweisungen, welche eine übersichtliche Programmierung gewährleisten.\\
Zu diesen zählen:
\begin{itemize}
	\item \textbf{Zuweisungen:} \texttt{Variable := Wert;}  
	Beispiel: \texttt{Motor\_Start := TRUE;}
	\item \textbf{Bedingte Anweisungen (IF-Strukturen):}  
	\texttt{IF Sensor = TRUE THEN Motor := TRUE; END\_IF;}
	\item \textbf{Mehrzweigige Bedingungen (IF-ELSIF-ELSE):}  
	\texttt{IF Temp > 50 THEN Alarm := TRUE; ELSIF Temp > 30 THEN Warning := TRUE; ELSE Alarm := FALSE; END\_IF;}
	\item \textbf{Schleifenstrukturen:}  
	\texttt{FOR i := 1 TO 5 DO Count := Count + 1; END\_FOR;}
	\item \textbf{Vergleiche und logische Operatoren:}  
	\texttt{AND, OR, NOT, >, <, =, >=, <=}
\end{itemize}
Diese Anweisungen ermöglichen selbst komplexe Programme zu realisieren. Es können mathematische Berechnungen deutlich einfacher durchgeführt werden als beispielsweise in der Programmiersprache \textbf{FUP}.
\subsubsection*{Vergleich SCL - FUP}
Die Programmiersprachen \textbf{FUP} und \textbf{SCL} werden beide für die Programmierung speicherprogrammierbarer Steuerungen verwendet.\\
Da \textbf{SCL} eine Siemens-eigene Sprache ist, konzentriert sich dieser Vergleich ausschließlich auf \textbf{STEP 7}.\\
Wie bereits erwähnt, ermöglicht \textbf{FUP} eine einfache blockbasierte Programmierung.\\
\textbf{FUP} eignet sich besonders für einfache Anwendungen, wie beispielsweise die Ansteuerung von Drehstrommotoren. Auch für einfache Schrittketten ist \textbf{FUP} gut geeignet.\\
Mit zunehmender Programmkomplexität leidet jedoch die Übersichtlichkeit, was zu einer erhöhten Fehleranfälligkeit führen kann.\\
Um diese Probleme zu vermeiden, wird in großen Produktionsanlagen mit komplexen Abläufen überwiegend textbasiert, also in \textbf{SCL}, programmiert.
Die folgenden Darstellungen vergleichen beide Sprachen und verdeutlichen ihre Unterschiede.
Dargestellt ist eine einfache Förderbandsteuerung in \textbf{FUP} und \textbf{SCL} im TIA Portal.
\begin{figure}[H]
	\centering
	\begin{subfigure}[b]{0.45\textwidth}
		\centering
		\includegraphics[width=0.8\linewidth]{images/PrgBeispiel_FUP}
		\caption{Funktionsbaustein zur Steuerung des Förderbands}
	\end{subfigure}
	\hfill
	\begin{subfigure}[b]{0.45\textwidth}
		\centering
		\includegraphics[width=0.8\linewidth]{images/PrgBeispiel_FUP_2}
		\caption{Freigabe im Haupt-OB}
	\end{subfigure}
	\caption{Realisierung der Förderbandsteuerung in \textbf{FUP}}
	\label{fig: Beispiel FUP}
\end{figure}
\newpage
\begin{figure}[H]
	\centering
	% Erste Spalte – Förderband-Baustein
	\begin{subfigure}[t]{0.4\textwidth}
		\centering
		\begin{lstlisting}[language=Pascal]
IF #OFF THEN
	#SR_Band := FALSE;	
ELSIF #ON THEN
	#SR_Band := TRUE;
END_IF;
			
#OU1 := #SR_Band AND #ENABLE;
		\end{lstlisting}
		\caption{SCL-Code des Förderband-Bausteins}
		\label{lst:Foerderband}
	\end{subfigure}
	\hfill
	% Zweite Spalte – Aufruf im Hauptprogramm
	\begin{subfigure}[t]{0.4\textwidth}
		\centering
		\begin{lstlisting}[language=Pascal]
IF NOT "S0_Stop" AND "S1_Anl_Ein" THEN
	"Pl_Freigabe" := TRUE;
ELSIF "S0_Stop" THEN
	"Pl_Freigabe" := FALSE;
END_IF;
			
"Foerderband"(
OFF    := "B2_vorne",
ON     := "B1_hinten",
ENABLE := "Pl_Freigabe",
OU1    => "Q1_Band");
		\end{lstlisting}
		\caption{Aufruf des Förderband-Bausteins im Hauptprogramm}
		\label{lst:Foerderband_call}
	\end{subfigure}
	\caption{Realisierung des Förderbandprogramms in \textbf{SCL}}
	\label{fig:Foerderband_Listing}
\end{figure}

Anhand des Beispiels (\ref{fig: Beispiel FUP} \ref{fig:Foerderband_Listing}) wird deutlich, dass \textbf{FUP} auf den ersten Blick leichter verständlich wirkt, da es sich um die grafische Verschaltung logischer Funktionen handelt.\\
Auf den zweiten Blick zeigt sich jedoch, dass der Programmieraufwand deutlich höher ist als in \textbf{SCL}. In \textbf{SCL} kann dieselbe Funktionalität mit wenigen, klar strukturierten Codezeilen realisiert werden, ohne das Programm mit zusätzlichen Netzwerken zu überladen.\\
Zusammenfassend lässt sich festhalten, dass sowohl \textbf{FUP} als auch \textbf{SCL} ihre jeweiligen Vor- und Nachteile besitzen. Je nach Anwendung und Komplexität des Prozesses fällt die Entscheidung für die eine oder die andere Programmiersprache.

\subsubsection{Industrielle Bussysteme}
In der Automatisierung sind Kommunikationsprotokolle entscheidend für die Vernetzung einzelner Systeme. Die Funktionalität einer automatisierten Anlage basiert auf der erfolgreichen Kommunikation der einzelnen Komponenten.
\subsubsection*{Bussysteme in der Automatisierungstechnik}
Ein großer Bestandteil im Bereich der Kommunikation zwischen verschiedenen Teilnehmern bilden sogenannte Bussysteme.\\
Sie kommen zum Einsatz, wenn eine Steuerung mit den Sensoren bzw. Aktoren eines Systems effizient kommunizieren soll. Die Steuerung kann z.\,B. eine \textbf{SPS} oder ein \textbf{Mikrocontroller} sein.\\
Früher wurden Signale hauptsächlich über Hartverdrahtung übertragen. Diese Methode war zwar zuverlässig, jedoch sehr aufwändig zu implementieren und mit hohem Material- und Zeitaufwand verbunden.\\
Deshalb wurden die bereits erwähnten Bussysteme eingeführt. Mit deren Hilfe ist es möglich, mit geringem Verdrahtungsaufwand mit mehreren Teilnehmern innerhalb eines Systems zu kommunizieren.\\
Ein Bussystem besteht grundsätzlich aus einer Leitung, über die mehrere Teilnehmer angesprochen werden können. Die Topologie ist hierbei vielfältig. Es gibt sogenannte \textbf{Stern}-, \textbf{Linien}- oder auch \textbf{Ring}-Topologien.
\begin{figure}[H]
	\centering
	\includegraphics[width=0.4\linewidth]{images/Topologien}
	\caption{Gängige Bus- Topologien; Quelle \cite{Bustopologien}}
	\label{Bustopologien}
\end{figure}
Die am häufigsten angewandte Methode ist die Kommunikation über eine \textbf{Linientopologie}.\\
Hierbei werden mehrere Teilnehmer über einzelne Abzweige nacheinander angesteuert.\\
Bussysteme können auf den verschiedensten Ebenen der Automatisierungspyramide\cite{IO_Link} zum Einsatz kommen. Der Haupteinsatz erfolgt jedoch auf der Feld- und Steuerungsebene.\\
Weiterhin gibt es verschiedene Arten von Bussystemen, die sich in ihrer Struktur, Geschwindigkeit und Übertragungsart unterscheiden.
Die am häufigsten verwendeten sind im Folgenden aufgelistet:
\begin{itemize}
	\item Feldbus-basierte Kommunikationssysteme (z.\,B. \textbf{Profibus})
	\item Ethernet-basierte Kommunikationssysteme (z.\,B. \textbf{PROFINET})
	\item Punkt-zu-Punkt-Systeme (z.\,B. \textbf{IO-Link})
\end{itemize}
Im Folgenden eine kurze Gegenüberstellung der einzelnen Protokolle:
\begin{table}[H]
	\centering
	\label{tab:busvergleich}
	\renewcommand{\arraystretch}{1.2}
	\begin{tabular}{|p{3.5cm}|p{3.5cm}|p{3.5cm}|p{3.5cm}|}
		\hline
		\textbf{Kriterium} & \textbf{IO-Link} & \textbf{PROFINET} & \textbf{PROFIBUS} \\ \hline
		\textbf{Kommunikations- prinzip}  & Punkt-zu-Punkt (PTP) & Ethernet-basiert & Feldbus-basiert \\ \hline
		\textbf{Topologie} & Baum & Linie, Stern, Ring & Linie, Baum \\ \hline
		\textbf{Kommunikations- rate} & 4{,}8 / 38{,}4 / 230{,}4\,kBaud & bis 100\,Mbit/s & bis 12\,Mbit/s \\ \hline
		\textbf{Typische Anwendung} & Sensorik, Aktorik, Handhabungstechnik & Steuerung, Kommunikation, Anlagenvernetzung & Maschinensteuer., Antriebsregelung \\ \hline
	\end{tabular}
	\caption{Vergleich der Kommunikationssysteme IO-Link, PROFINET und PROFIBUS; Quellen: \cite{IO_Link}\cite{Schnell.2019}\cite{Langmann.2021}}
\end{table}
\subsubsection{IO-Link}
IO-Link ist ein Kommunikationssystem, welches der Anbindung intelligenter Sensoren und Aktoren dient.\\
Es wurde entwickelt, um die Lücke zwischen Feldbussen und Industrial-Ethernet-Systemen zu schließen. Der Standard ist in der Norm IEC~61131-9 definiert. IO-Link wird häufig in Fabrikautomatisierungsanlagen eingesetzt.
\subsubsection*{Aufbau und Funktionsweise von IO-Link}
IO-Link ist ein Kommunikationssystem, das nach dem \textbf{Point-to-Point} (PTP)-Prinzip arbeitet. Das bedeutet, es werden mehrere Teilnehmer Punkt zu Punkt angesteuert. Der \textbf{IO-Link-Master} ist hierbei mit den \textbf{IO-Link-Devices} PTP-verbunden. Dieser stellt die Schnittstelle zwischen der Steuerung (\textbf{SPS}) und der Sensor-/Aktor-Ebene dar.
\begin{figure}[H]
	\centering
	\includegraphics[width=0.7\linewidth]{images/IO_Link_Architektur}
	\caption{Architektur IO\_Link; Quelle \cite{IO_Link}}
	\label{IO_Link_Architektur}
\end{figure}
Das System arbeitet \textbf{bidirektional}\cite{IO_Link}. Das bedeutet, der IO-Link-Master kann sowohl Prozessdaten empfangen als auch neue Parameter an die Devices senden (z.\,B. \textbf{RFID}).\\
Die Verbindung erfolgt über dreipolige Standardleitungen\ref{IO_Link_Anschluss}. Dadurch lässt sich IO-Link problemlos in das Gesamtsystem integrieren. Eine separate Busverkabelung ist nicht notwendig\cite{isa_iolink_architecture_2016}.
\begin{figure}[H]
	\centering
	\includegraphics[width=0.7\linewidth]{images/IO_Link_3Polig}
	\caption{Anschlussbelegung IO-Link; Quelle \cite{IO_Link}}
	\label{IO_Link_Anschluss}
\end{figure}
Der IO-Link-Master\ref{IO_Link_Master} kann über ein übergeordnetes Bussystem (z.\,B. \textbf{Profibus}, \textbf{PROFINET}) mit der SPS kommunizieren\cite{Langmann.2021}.
\begin{figure}[H]
	\centering
	\includegraphics[width=0.4\linewidth]{images/AL1100}
	\caption{Master IO\_Link; Quelle \cite{IFM_IO_Link_Master}}
	\label{IO_Link_Master}
\end{figure}
\subsubsection*{Kommunikation und Datenaustausch}
Die Kommunikation über IO-Link erfolgt seriell. Hierbei wird mit festen Baudraten von 4{,}8\,kBaud, 38{,}4\,kBaud oder 230{,}4\,kBaud übertragen.\\\\
Die Datenkommunikation gliedert sich in drei Hauptarten\cite{IO_Link}:
\begin{itemize}
	\item \textbf{Prozessdaten}:\\
	Diese werden zyklisch zwischen Master und Device ausgetauscht. Sie enthalten z. B. Messwerte von Sensoren oder Schaltzustände von Aktoren.
	\item \textbf{Servicedaten (Parameterdaten)}:\\
	Sie werden azyklisch übertragen und dienen zur Parametrierung oder Konfiguration der Geräte. So können beispielsweise Schaltpunkte, Messbereiche oder Filterzeiten direkt über die Steuerung angepasst werden.
	\item \textbf{Ereignisdaten}:\\
	Hierbei handelt es sich um Status- oder Fehlermeldungen, die das Device bei bestimmten Zuständen (z. B. Übertemperatur, Kabelbruch oder Spannungsfehler) an den Master sendet.
\end{itemize}
Durch diese Aufteilung ist eine gezielte Überwachung, Diagnose bei Fehlern und Konfiguration der einzelnen Module möglich. Eine Prozessunterbrechung ist hierfür nicht notwendig.\\
Auf Protokollebene wird zwischen drei Schichten unterschieden\cite{IO_Link}:
\begin{itemize}
	\item \textbf{Physikalische Schicht} – definiert die elektrische Verbindung über die dreipolige Leitung.
	\item \textbf{Datenverbindungsschicht} – regelt das Telegrammformat und die Fehlererkennung.
	\item \textbf{Applikationsschicht} – beschreibt den Austausch von Prozess-, Service- und Ereignisdaten.
\end{itemize}
\subsubsection*{Komponenten eines IO-Link-Systems}
Ein vollständiges IO-Link-System besteht in der Regel aus folgenden Komponenten\cite{IO_Link}:
\begin{itemize}
	\item \textbf{IO-Link-Master}:\\
	Er verwaltet die Kommunikation zu den einzelnen IO-Link-Devices und bildet die Schnittstelle zur übergeordneten Steuerung.
	Je nach Ausführung kann der Master als Schaltschrankmodul oder als Feldmodul ausgeführt sein. Letzteres wird direkt in der Anlage montiert, wodurch sich Installationsaufwand und Verkabelungslänge reduzieren.
	\item \textbf{IO-Link-Devices}:\\
	Hierbei handelt es sich um Sensoren, Aktoren oder andere Feldgeräte, die über IO-Link kommunizieren. Typische Beispiele sind Drucksensoren, Wegmesssysteme, Ventilinseln, RFID-Leseeinheiten oder Smart Lights.
	Jedes Device verfügt über eine IODD-Datei (\textit{IO Device Description}), in der alle Geräteparameter, Kommunikationsdaten und Diagnosemöglichkeiten beschrieben sind. Diese Datei wird im Engineering-Tool der Steuerung verwendet, um das Gerät automatisch zu erkennen und korrekt zu parametrieren.
	\item \textbf{Verkabelung}:\\
	IO-Link nutzt standardisierte, ungeschirmte Leitungen mit M12- oder M8-Steckverbindern. Die Kabellängen betragen typischerweise bis zu 20\,m zwischen Master und Device. Da das Signal digital übertragen wird, ist die Verbindung unempfindlich gegenüber elektromagnetischen Störungen und Spannungsabfällen.
\end{itemize}
\subsubsection*{Zusammenfassung}
Über IO-Link ist eine Kommunikation bis in die unterste Feldebene möglich. Analoge Signale sind hierbei überflüssig, da die Schnittstelle standardisiert ist. Weiterhin lassen sich Geräte identifizieren und parametrieren. Außerdem stehen umfassende Diagnosemöglichkeiten zur Verfügung.\\
Damit trägt IO-Link wesentlich zur Transparenz, Flexibilität und Wartungsfreundlichkeit moderner Automatisierungssysteme bei und bildet eine wichtige Grundlage für Konzepte wie \textbf{Industrie~4.0}\cite{IO_Link}.
\subsubsection{Vorteile und Nutzen in der Praxis}
Der Einsatz von IO-Link in der Industrie bringt zahlreiche Vorteile mit sich. Es werden technische Defizite von klassischen Bus- und Anschlusssystemen eliminiert. Neben der Vereinfachung der Verdrahtung bietet das System eine hohe Flexibilität, verbesserte Diagnosefähigkeiten sowie eine sehr gute digitale Kommunikation bis in die Feldebene hinein.\\
Durch die genannten Punkte wird nicht nur die Leistungsfähigkeit der Anlage gesteigert, sondern auch die Wirtschaftlichkeit und Nachhaltigkeit in einer Produktion verbessert.
\subsubsection*{Technische Vorteile}
Ein zentraler technischer Vorteil von IO-Link besteht in der vollständig digitalen Signalübertragung zwischen dem IO-Link-Master und den angeschlossenen Devices. Dadurch entfallen typische Nachteile analoger Schnittstellen wie Messfehler durch Spannungsabfall, Rauschen oder Kalibrierungsabweichungen. Jeder über IO-Link verbundene Sensor oder Aktor überträgt exakte Prozessdaten in digitaler Form, wodurch die Messqualität und Zuverlässigkeit deutlich erhöht werden.\\
Ein weiterer Vorteil ist die sogenannte bidirektionale Kommunikation. Hierbei ist es möglich, dass der IO-Link-Master Daten empfangen kann, sowie in der Lage ist, die Parameter der Sensoren oder Aktoren zu beschreiben.\\
Darüber hinaus bietet IO-Link standardisierte Schnittstellen und Datenstrukturen, die herstellerübergreifend kompatibel sind. Jedes Gerät wird anhand seiner IODD-Datei (IO Device Description) automatisch erkannt, was die Integration in bestehende Systeme erheblich vereinfacht\cite{IO_Link}. Diese Einheitlichkeit führt zu einer hohen Austauschbarkeit der Komponenten, ohne dass Änderungen in der SPS-Software oder in der Verdrahtung notwendig sind.
\subsubsection{Weitere Vorteile}
Durch den verringerten Verdrahtungsaufwand können erhebliche Kosten eingespart werden. Diese können an anderer Stelle effizienter eingesetzt werden.\\
Da ein IO-Link-Master in der Lage ist, sowohl klassische als auch digitale Signale zu verarbeiten, entfallen zusätzliche Baugruppen zur Verarbeitung bestimmter Signalarten. Diese Punkte führen zu einer Erhöhung bzw. Verbesserung der Wirtschaftlichkeit. Dadurch kann nachhaltiger und effektiver gearbeitet werden.
\subsubsection*{Erweiterte Diagnosemöglichkeiten}
Eine besondere Eigenschaft von IO-Link liegt in den umfassenden Diagnosemöglichkeiten, die dem Anwender zur Verfügung stehen. Jedes \textbf{Device} ist in der Lage, sich selbst zu diagnostizieren und die entsprechenden Informationen an den \textbf{IO-Link-Master} zu übermitteln.\\
Somit werden Fehler wie Kabelbruch, Kurzschluss oder Unterspannung sofort an die Steuerung gemeldet. Dadurch ist ein schnelles und effektives \textit{Trouble Monitoring} gegeben. Dies führt dazu, dass langfristig Fehlerquellen und Anfälligkeiten in Systemen erkannt und gezielt beseitigt werden können.
\subsubsection{Praxisbeispiel}
In der industriellen Praxis findet IO-Link zunehmend Anwendung bei intelligenten Sensoren und Aktoren, etwa in Montage- und Prüfprozessen oder bei der Handhabungstechnik.\\
Ein typisches Beispiel ist die Integration eines IO-Link-Füllstandssensors oder Drucksensors in ein SPS-System. Die Messwerte werden dabei digital und störungsfrei übertragen, während gleichzeitig Geräteparameter über die Steuerung angepasst werden können – etwa der Schaltpunkt oder der Messbereich\cite{IO_Link}.
\subsubsection{Zusammenfassung}
IO-Link vereint einfache Installation mit intelligenter Kommunikation. Die Technologie schafft eine durchgängige Datenbasis von der Feldebene bis zur Steuerung und ermöglicht so eine effiziente, flexible und zukunftssichere Automatisierung.\\
Durch die Kombination aus digitaler Signalübertragung, erweiterter Diagnose und automatischer Parametrierung wird IO-Link zu einem zentralen Bestandteil moderner Industrie-4.0-Architekturen.
\subsubsection{Sicherheit in der Automatisierung}
Im Kontext der Automatisierung von technischen Anlagen ist die Sicherheit der größte und wichtigste Aspekt.\\
Beim Betreiben eines Systems muss vollständig ausgeschlossen sein, dass keine Gefahr für Menschen, von der Anlage, ausgeht. Das bedeutet, bei Planung der Anlage, steht die Sicherheit im Fokus.\\
Speziell in der Produktion bzw. in der Logistik sind sicherheitstechnische Aspekte enorm wichtig.\\
Bei automatisierten Prozessen wie z.B. das Verfahren eines Hochregallagers, können Gefahren durch Implementierung von Sicherheitslogiken ausgeschlossen werden.\\
Das Ziel ist immer Gefährdungen bzw. Verletzungen von Menschen zu verhindern.
\subsubsection*{Einordnung in den Kontext der funktionalen Sicherheit}
Um die Sicherheitsprinzipien besser zu verstehen ist es wichtig die einzelnen Aspekte der Sicherheitsthematik in den Gesamtkontext einzuordnen.\\
In diesem Zusammenhang spielt der Bereich funktionale Sicherheit eine entscheidende Rolle.\\
Hierunter versteht man den Teil einer Maschine der von den korrekt funktionierenden Steuerungs- und Sicherheitskomponenten abhängt. Sie sorgt dafür, das bei kritischen Situationen ein sicherer Zustand der Anlage erreicht wird. EIn Beispiel hierfür wäre das direkte Abschalten eines Motors.\\
Damit sicherheitsbezogene Steuerungssysteme zuverlässig arbeiten, müssen sie bestimmte Anforderungen erfüllen, die in internationalen Normen festgelegt sind.\\
Einige dieser Vorschriften sind im folgenden aufgelistet:
\begin{itemize}
	\item \textbf{EN ISO 13849-1}: Beschreibt die sicherheitsrelevanten Teile einer Anlage, u.A. \textbf{PL (Performance Level)}
	\item \textbf{IEC 62061}: Behandelt die funktionale Sicherheit elektrischer, elektronischer und programmierbarer Steuerungssysteme. Baisert auf \textbf{SIL} (Safety Integrity Level )
	\item
\end{itemize}


\section{Implementierung eines Lichtgatters in das Gesamtkonzept}
Im aktuellen Ist-Zustand des Hochregallagers fehlt die vollständige Implementierung der Sicherheitslogik.\\
Lediglich der Not-Aus-Schalter ist hardwareseitig verdrahtet und schaltet bei Betätigung die Zuleitung ab. Es existiert jedoch keine automatische Abschaltung im Gefahrenfall.\\
Die Logik des Ein- und Auslagerns ist somit nicht gefahrenfrei. Sobald ein Eingriff in den Prozess erfolgt, kann dieser im aktuellen Zustand nicht automatisch gestoppt werden.\\
Da das Miniaturmodell eine reale Situation in der Logistik abbilden soll, stellt dieser Punkt ein erhebliches Defizit dar.\\
Es gibt verschiedene Möglichkeiten, um diesen Nachteil zu beheben. Die naheliegendste Option ist die Implementierung einer \textbf{Lichtgatter-Funktion}, um den Gefahrenbereich zuverlässig zu überwachen und den Betrieb im Störfall sicher zu unterbrechen.
Durch ein vorheriges Projekt wurde bereits, ein Modell der Firma IFM\cite{Lichtgatter_ifm}, installiert.
\subsection{Zielsetzung der Implementierung}
Die Studienarbeiten der vergangenen Jahre haben das Hochregallager in einen Zustand gebracht, der wichtige Funktionen wie Ein- und Auslagern abbildet.\\
Jedoch wurden hierbei, fälschlicherweise, die notwendigen Sicherheitsaspekte völlig außer Acht gelassen.\\
In einer realen Umgebung, beispielsweise in Materiallagern mit Hochregalsystemen großer Industrieunternehmen, wäre eine solche Vorgehensweise undenkbar.\\
Es muss jederzeit gewährleistet sein, dass Gefahren so gut wie möglich unter Kontrolle gehalten werden. Das bedeutet, im Fehlerfall muss jede gefährliche Bewegung sicher abgeschaltet werden.\\
Dies stellt die zentrale Zielsetzung der Implementierung dar. In jedem Betriebsmodus muss sichergestellt sein, dass bei Erkennung einer Gefahr automatisch abgeschaltet wird.\\
Dabei ist jedoch zu berücksichtigen, dass der Kolben des Zylinders, welcher die Materialien einlagert, kein Abschaltsignal auslösen darf. Dieser fährt nämlich unmittelbar in den Gatterbereich der Lichtschranke ein.\\
Somit ist zunächst eine umfassende Analyse des Ist-Zustands sowie der verschiedenen Prozesswerte des Lichtgatters erforderlich.\\
Erst im Anschluss kann die Implementierung der Sicherheitslogik erfolgen.

\subsection{Analyse der Parametrierung des Lichtgatters}
\label{cha:Lichtgatter}
Die Konfiguration des Lichtgatters ist entscheidend für die richtige Funktionsweise. Beim Modell von \textbf{IFM}\cite{Lichtgatter_ifm} gibt es mehrere Möglichkeiten zur Parametrierung.\\
Das Lichtgatter besitzt mehrere Lichtstrahlen, welche über einzelne Bits definiert sind. Sobald sich ein Objekt im Gatter befindet, werden mehrere dieser Lichtstrahlen gleichzeitig unterbrochen.\\
Der Sensor des vorliegenden Lichtgatters hat im sogenannten \textbf{Detektionsmodus (GTBO)} den Wert 1. Das bedeutet, dass er schaltet, sobald ein Objekt die Strahlen unterbricht.\\
Über \textbf{IO-Link} kann nun festgelegt werden, ab wie vielen unterbrochenen Strahlen der Sensor ein Schaltsignal ausgeben soll.
\subsubsection{Prozesswerte zur Steuerung des Arbeitsverhaltens}
Zur Steuerung des Verhaltens bei einer Unterbrechung stehen verschiedene Prozesswerte zur Verfügung\cite{Lichtgatter_ifm_2}:
\begin{itemize}
	\item \textbf{FBO (First Beam Occupied)}
	\item \textbf{LBO (Last Beam Occupied)}
	\item \textbf{CBO (Central Beam Occupied)}
	\item \textbf{NBO (Number of Beams Occupied)}
	\item \textbf{NCBO (Number of Consecutive Beams Occupied)}
\end{itemize}

Im Betriebsmodus \textbf{FBO} schaltet der Sensor bereits beim ersten unterbrochenen Lichtstrahl. Das bedeutet, Objekte, die im unteren Bereich des Lichtgatters liegen, werden als erstes erfasst.\\
Im Modus \textbf{LBO} geschieht genau das Gegenteil: Es wird beim letzten Lichtstrahl geschaltet.\\
Dieser Modus wird beispielsweise bei Endlagenerkennungen oder beim Auslagern verwendet – also immer dann, wenn das Austreten aus einem bestimmten Bereich überwacht werden soll.\\

Der Modus \textbf{CBO} ist etwas komplexer:
Hier wird der mittlere Lichtstrahl (bei mehreren zusammenhängenden Strahlen) überwacht. Tritt ein Objekt in das Lichtgatter ein, wird der zentrale Strahl als Referenzpunkt betrachtet. Bei mehreren Objekten wird immer vom größeren Objekt ausgegangen, also von dem, das mehr Strahlen gleichzeitig unterbricht – wie in Abbildung \ref{CBO_Modus} zu sehen.\\
Anwendungsbeispiele sind z. B. Durchgänge oder Förderabschnitte, also überall dort, wo Objekte mittig erfasst werden sollen (Referenzpunkt: Mitte).

\begin{figure}[H]
	\centering
	\includegraphics[width=0.7\linewidth]{images/CBO_Modus}
	\caption{Modus \textbf{CBO}; Quelle: \cite{Lichtgatter_ifm_2}}
	\label{CBO_Modus}
\end{figure}

Die letzten beiden Modi, \textbf{NBO} und \textbf{NCBO}, beschäftigen sich mit der konkreten Anzahl der Lichtstrahlen, die unterbrochen werden.\\
Der Unterschied der beiden Modi liegt darin, dass \textbf{NBO} die Gesamtanzahl der unterbrochenen Lichtstrahlen angibt, während \textbf{NCBO} die Anzahl der Strahlen beschreibt, die \textit{aufeinanderfolgend} unterbrochen werden.\\
Im Projekt wird der Modus \textbf{NBO} verwendet. Eine Abbildung soll die Funktionsweise noch einmal verdeutlichen (siehe Abbildung \ref{NBO_Modus}).
\begin{figure}[H]
	\centering
	\includegraphics[width=0.7\linewidth]{images/NBO_Modus}
	\caption{Funktionsweise des \textbf{NBO-Modus}; Quelle: \cite{Lichtgatter_ifm_2}}
	\label{NBO_Modus}
\end{figure}

\subsection{Komponentenübersicht und Systemaufbau}
Für die Umsetzung der Sicherheitsfunktion wurde ein Systemaufbau realisiert, der die Lichtschranke direkt mit einem IO-Link-Master und der SPS verbindet. Im Folgenden werden die eingesetzten Komponenten sowie deren Zusammenspiel innerhalb des Gesamtsystems beschrieben.\\[0.3cm]

\begin{table}[H]
	\centering
	\renewcommand{\arraystretch}{1.3}
	\begin{tabular}{|p{4cm}|p{4cm}|p{5cm}|p{2.5cm}|}
		\hline
		\textbf{Komponente} & \textbf{Hersteller / Typ} & \textbf{Funktion} & \textbf{Schnittstelle} \\ \hline
		Lichtgitter / Lichtschranke & IFM OY5100 (Infrarot) & Erfassung von Objekten und Überwachung des Gefahrenbereichs & IO-Link \\ \hline
		IO-Link Master & IFM AL1100 & Kommunikation zwischen IO-Link-Devices und der SPS & PROFINET / IO-Link \\ \hline
		SPS & Siemens S7-1500 & Steuerung des gesamten Systems, Verarbeitung der Sensorsignale & PROFINET \\ \hline
		Netzteil 24V DC & Siemens SITOP / vergleichbar & Versorgung von IO-Link-Master, Lichtschranke und SPS & 24V DC \\ \hline
		Aktorik (Zylinder / Motor) & Festo (pneumatisch) & Ausführung der Ein- und Auslagerbewegung & Digitale Ausgänge der SPS \\ \hline
	\end{tabular}
	\caption{Übersicht der verwendeten Komponenten im Systemaufbau}
	\label{tab:komponenten}
\end{table}

In Tabelle~\ref{tab:komponenten} sind die einzelnen Komponenten des Hochregallagers rund um die Lichtschranke dargestellt.\\
Diese müssen optimal miteinander interagieren, um die gewünschte Funktionalität sicherzustellen.\\
Die Lichtschranke ist vom Typ \textbf{OY5100} und stammt vom Hersteller \textbf{IFM}. Es handelt sich hierbei um ein steuerbares Lichtgitter, das über IO-Link mit der SPS kommuniziert.
\subsubsection{Systemaufbau}
Die folgende Abbildung zeigt den strukturellen Aufbau des Gesamtsystems in vereinfachter Form.\\[0.2cm]

\begin{figure}[H]
	%\hspace*{-1cm} % optional: Diagramm leicht nach links verschoben
	\begin{tikzpicture}[
		node distance=1.8cm,
		>=latex,
		font=\small,
		every node/.style={align=center}
		]
		% Nodes (vertikal angeordnet)
		\node (sps) [draw, rectangle, fill=orange!10, minimum width=3.5cm, minimum height=1cm] {SPS\\(Siemens S7-1500)};
		\node (master) [draw, rectangle, fill=green!10, below=of sps, minimum width=3.5cm, minimum height=1cm] {IO-Link Master\\(IFM AL1100)};
		\node (sensor) [draw, rectangle, fill=blue!10, below=of master, minimum width=3.5cm, minimum height=1cm] {Lichtgitter\\(IFM OY5100)};
		\node (actuator) [draw, rectangle, fill=red!10, below=of sensor, minimum width=3.5cm, minimum height=1cm] {Aktorik\\(Zylinder / Motor)};
		\node (power) [draw, rectangle, fill=gray!15, left=3cm of master, minimum width=3cm, minimum height=1cm] {Netzteil\\24V DC};
		
		% Signalfluss (von oben nach unten)
		\draw[->, thick] (sps) -- (master) node[midway, right]{\textbf{PROFINET}};
		\draw[->, thick] (master) -- (sensor) node[midway, right]{\textbf{IO-Link}};
		\draw[->, thick] (sensor) -- (actuator) node[midway, right]{\textbf{Signal / Reaktion}};
		
		% Spannungsversorgung (seitlich)
		\draw[->, thick] (power) -- (master) node[midway, above]{24V DC};
		\draw[->, thick] (power) |- (sensor);
		\draw[->, thick] (power) |- (sps);
		
	\end{tikzpicture}
	\caption{Vertikaler Systemaufbau des Lichtgatter-Systems}
	\label{fig:systemaufbau_vertikal}
\end{figure}

Im Diagramm~\ref{fig:systemaufbau_vertikal} ist der Aufbau der einzelnen Komponenten grafisch dargestellt. Der IO-Link-Master dient hierbei als Schnittstelle zwischen SPS und Lichtgitter und steuert die Kommunikation über das im TIA Portal implementierte Programm.\\
Im Fehlerfall werden gefährliche Bewegungen, die beispielsweise vom Zylinder oder den Motoren der Linearachsen ausgehen, automatisch gestoppt.\\
Im folgenden Abschnitt wird die Implementierung dieser Sicherheitslogik in der SPS detailliert erläutert.

\subsection{Implementierung der Sicherheitslogik in der SPS}
Die Sicherheitslogik wird, wie bereits erwähnt, in eine \textbf{Siemens S7-1500} SPS einprogrammiert. Die Programmierung erfolgt mithilfe der Software \textbf{TIA Portal}.\\
Für die Umsetzung der Arbeit wurde das bestehende Programm des Jahrgangs \textbf{TEA 22} als Grundlage verwendet. Zunächst wurde eine neue \textit{Program Organization Unit (POU)} mit der Bezeichnung \textbf{Lichtschranke} erstellt.\\
Im Anschluss erfolgte das Einbinden des Moduls \textbf{OY5100} im Reiter \textbf{Geräte und Netzwerke}. Der Adressbereich \textit{(frei wählbar)} wurde so konfiguriert, dass eine eindeutige Zuordnung innerhalb des Projekts gewährleistet ist.\\[0.2cm]

\begin{figure}[H]
	\centering
	\includegraphics[width=0.8\linewidth]{images/OY5100_Einbindung} % Beispielabbildung
	\caption{Einbindung der IFM-Lichtschranke OY5100 im TIA Portal}
	\label{fig:tia_device_network}
\end{figure}

Die zugewiesenen Adressen werden für die verschiedenen Prozesswerte genutzt (siehe Kapitel~\ref{cha: Prozesswerte}).\\
Als Vorbereitung auf die Programmierung wurde eine sogenannte \textit{Watchtable} erstellt. Diese dient dazu, die einzelnen Prozesswerte des Lichtgatters während der Laufzeit zu überwachen und deren Verhalten zu analysieren.\\
Die Prozesswerte liegen im Format \textbf{WORD} (2 Byte) vor und bilden die Anzahl der unterbrochenen Lichtstrahlen ab.\\[0.2cm]

Nach einer Reihe von Tests wurde festgestellt, dass der Kolben des Zylinders im Arbeitsbereich des Lichtgatters \textbf{drei Strahlen} gleichzeitig unterbricht. Hierfür erfolgte ein gezieltes Verfahren der Lineareinheit über alle vier Ebenen des Hochregallagers. So konnte überprüft werden, dass der Kolben in jeder Position konstant drei Strahlen abdeckt.\\[0.2cm]

Im weiteren Verlauf wurde diskutiert, welcher Prozesswert sich für die Sicherheitslogik am besten eignet. Es zeigte sich, dass der Modus \textbf{NBO (Number of Beams Occupied)} die Summe aller unterbrochenen Lichtstrahlen angibt (siehe Kapitel~\ref{cha: Prozesswerte}). Daher ist dieser Prozesswert optimal geeignet, um eine zuverlässige Abschaltung über das Lichtgatter zu realisieren.\\
Nach der Diskussion wurde, für die Realisierung der Abfrage, ein neuer FB (Freigabe\_Lichtschranke\_FB24) aufgebaut in \textbf{SCL}. Im Diagramm \ref{fig:fb24_einbindung} ist der Zusammenhang zwischen den einzelnen Modi und der \textbf{Freigabe} dargestellt.
\begin{figure}[H]
	\hspace*{3cm}
	\begin{tikzpicture}[
		node distance=4cm and 4cm,
		>=latex,
		font=\small,
		every node/.style={align=center}
		]
		
		% Zentrale Freigabe
		\node (fb24) [draw, circle, fill=blue!10, minimum width=2.8cm, align=center, thick] 
		{FB24\\Freigabe\\Lichtschranke};
		
		% Außenliegende FBS
		\node (fbhand) [draw, rectangle, fill=green!10, above=of fb24, minimum width=3.8cm, minimum height=1cm, thick]
		{FB\\Handbetrieb};
		\node (fbprozess) [draw, rectangle, fill=orange!10, right=of fb24, minimum width=4.3cm, minimum height=1cm, thick]
		{FB\\Prozesssteuerung\\(Ein-/Auslagerung)};
		\node (fbstoer) [draw, rectangle, fill=red!10, below=of fb24, minimum width=3.8cm, minimum height=1cm, thick]
		{FB\\Störungen};
		
		% Verbindungspfeile
		\draw[<->, thick] (fbhand.south) -- (fb24.north) node[midway, right]{Signalaustausch};
		\draw[<->, thick] (fbprozess.west) -- (fb24.east) node[midway, above]{Status / Freigabe};
		\draw[<->, thick] (fbstoer.north) -- (fb24.south) node[midway, right]{Rückmeldung / Fehler};
		
	\end{tikzpicture}
	\caption{Einbindung des Bausteins \texttt{FB24 – Freigabe\_Lichtschranke} in das Gesamtsystem}
	\label{fig:fb24_einbindung}
\end{figure}
Der Baustein \texttt{FB24 – Freigabe\_Lichtschranke} ist ein zentraler Bestandteil der Sicherheitsarchitektur.\\
Ohne ihn wäre ein sicheres Abschalten im Fehlerfall nicht möglich. Die \textbf{Prozesssteuerung} wird durch die Freigabe in jedem Schritt überwacht.\\
Durch die Einbindung in den \texttt{FB11 – Störungen} erfolgt eine Überwachung sämtlicher Ausgänge über die Variable \texttt{b\_Freigabe}. Diese wird gesetzt, sobald keine Störungen aktiv sind und die Freigabe durch das Lichtgitter gegeben ist.\\[0.2cm]

\begin{figure}[H]
	\centering
	\includegraphics[width=0.7\linewidth]{images/b_Freigabe}
	\caption{Setzen der Variable \texttt{b\_Freigabe} in \texttt{FB11 – Störungen}}
	\label{b_Freigabe}
\end{figure}
Jene Variable in Abbildung \ref{b_Freigabe} wurde bereits in vorherigen Projekten verwendet.\\
Sie wird im Block \texttt{FB\_Ausgänge} eingesetzt, als zusätzliche invertierte Rücksetzbedingung. Somit ist ein fehlerfreier Betriebsmodus der einzelnen Aktoren gewährleistet.
\subsubsection{Logische Abschaltbedingung der Lichtschranke}
\begin{figure}[H]
	\centering
	\begin{tikzpicture}[
		node distance=1.8cm,
		>=latex,
		font=\small,
		every node/.style={align=center},
		process/.style={draw, rectangle, rounded corners, fill=blue!10, minimum width=4cm, minimum height=1cm},
		decision/.style={draw, diamond, aspect=2, fill=orange!10, text width=3.8cm, align=center, inner sep=1pt},
		startstop/.style={draw, ellipse, fill=gray!15, minimum width=3cm, minimum height=1cm}
		]
		
		% Nodes
		\node (start) [startstop] {Start / Normalbetrieb};
		\node (read) [process, below=of start] {Prozesswert\\\texttt{NBO} auslesen};
		\node (decision) [decision, below=of read] {Ist \texttt{NBO > 3}?\\(3 = Zylinderhöhe)};
		\node (ok) [process, below left=1.5cm and 2.5cm of decision, fill=green!10] {Freigabe aktiv\\\texttt{b\_Freigabe = TRUE}};
		\node (stop) [process, below right=1.5cm and 2.5cm of decision, fill=red!10] {Aktorik abschalten\\\texttt{b\_Freigabe = FALSE}};
		\node (end) [startstop, below=of ok] {Zyklus fortsetzen};
		
		% Connections
		\draw[->, thick] (start) -- (read);
		\draw[->, thick] (read) -- (decision);
		\draw[->, thick] (decision) -- node[midway, left]{Nein} (ok);
		\draw[->, thick] (decision) -- node[midway, right]{Ja} (stop);
		\draw[->, thick] (ok) -- (end);
		
		
	\end{tikzpicture}
	\caption{Abschaltlogik der Lichtschranke basierend auf dem Prozesswert \texttt{NBO}}
	\label{fig:abschaltlogik_nbo}
\end{figure}
In Abbildung~\ref{fig:abschaltlogik_nbo} ist der Ablauf der Sicherheitsabschaltung grafisch dargestellt.
Sobald der Prozesswert \textbf{NBO} größer als 3 ist, wird die Aktorik abgeschaltet.
Das bedeutet, dass eine sofortige Abschaltung erfolgt, sobald zusätzlich zum Zylinder eine Hand oder ein anderes Objekt detektiert wird – also ein weiterer Lichtstrahl unterbrochen ist.

\newpage
\printbibliography
\end{document}