% Kapitel Einleitung

\section{Einleitung}
\label{cha: Einleitung}
\subsection{Motivation}
In Zeiten von \textbf{Industrie 4.0} hat die Bedeutung des Automatisierungsgrades in den vergangenen Jahren erheblich zugenommen.\\
Besonders im Hinblick auf Effizienz, Sicherheit und die Effektivität automatisierter Abläufe spielt die Automatisierung eine zentrale Rolle.\\
Der Begriff \textbf{Industrie 4.0} beschreibt den aktuellen Wandel in der industriellen Produktion hin zu intelligent vernetzten Systemen. Dabei spielt die Automatisierung eine entscheidende Rolle. Die Kommunikation der einzelnen Maschinen, Produktionsanlagen und Lager erfolgt hierbei über \textbf{industrielle Netzwerke}, was zu einem echtzeitfähigen Verhalten beitragen kann.\\
Durch die zunehmende Digitalisierung und Vernetzung industrieller Prozesse entstehen stetig neue Anforderungen an die Steuerungs- und Regelungstechnik. Diese Entwicklungen führen zu immer komplexeren, aber auch effizienteren Produktions- und Logistiksystemen.\\\\
Gerade im Bereich der Lagerlogistik ist es entscheidend, sämtliche Sicherheitsmaßnahmen einzuhalten, um einen reibungslosen und sicheren Ablauf zu gewährleisten. Gleichzeitig wird eine präzise und fehlerarme Handhabung von Materialien ermöglicht, was wesentlich zur Prozesssicherheit und Produktqualität beiträgt. Lagerprozesse sind ein wesentlicher Bestandteil des Materialflusses innerhalb der Produktion. Fehler in diesem Bereich können zu Stillständen und hohen Kosten führen. Um solche Probleme zu vermeiden, ist ein reibungsloser Ablauf aller Prozessschritte erforderlich.
\subsection{Bedeutung der Lagerlogistik}
Im Bereich der Automatisierungstechnik existieren zahlreiche Konzepte, um Produkte und Ersatzteile effizient zu lagern und zu verwalten.\\
Eine klassische Variante stellt die manuelle Lagerung in großen Lagerräumen dar. Dabei sind jedoch umfangreiche Such- und Zuordnungsprozesse erforderlich, um die gewünschten Komponenten zu identifizieren und zu entnehmen.\\
Eine wesentlich effektivere Lösung bietet die automatisierte Lagertechnik, insbesondere das sogenannte \enquote{Hochregallager}.\\
Ein Hochregallager ermöglicht die automatisierte Ein- und Auslagerung von Materialien und Ersatzteilen. Solche Systeme ermöglichen es durch den Einsatz moderner Sensorik, wie etwa der Materialerkennung über \textbf{RFID} oder der automatischen Auftragsausführung mittels QR-Code-Erkennung, Materialien präzise zu lagern und zu verwalten. Des Weiteren können die Prozesse über geeignete Softwarelösungen überwacht und gesteuert werden.\\
Dadurch können Prozesszeiten verkürzt, Fehlerquoten reduziert und die Übersichtlichkeit innerhalb des Lagers deutlich verbessert werden.\\\\
Grundsätzlich lässt sich die Handhabung in zwei Varianten unterteilen:
\begin{itemize}
	\item Automatische Ein- und Auslagerung
	\item Manuelle Ein- und Auslagerung
\end{itemize}
Das Hochregallager ist in der Regel in einem separaten Bereich installiert, in dem autorisierte Mitarbeiter mithilfe eines Auftrags- oder Identifikationssystems auf die eingelagerten Komponenten zugreifen können.\\
Die Automatisierung solcher Systeme ermöglicht eine zuverlässige, reproduzierbare und sichere Abwicklung logistischer Prozesse. Hierdurch werden menschliche Fehler weitestgehend reduziert und somit die Sicherheit erhöht.
\subsection{Das DHBW-Hochregallager-Modell}
Zur Simulation und Demonstration der grundlegenden Lagerfunktionen wurde an der Dualen Hochschule Baden-Württemberg (DHBW) eine Miniaturversion eines Hochregallagers entwickelt, die die wesentlichen Prozesse der Ein- und Auslagerung abbildet.\\
Das Modell eignet sich hervorragend als Lern- und Übungsplattform für Studierende, um praxisnah Kenntnisse in der industriellen Automatisierung zu erwerben. Ziel des Systems ist es, die Zusammenarbeit der einzelnen Sensoren und Aktoren zu simulieren und dadurch den Studierenden einen praxisnahen Einblick in die industrielle Automatisierung zu bieten.\\\\
Das System wurde in den vergangenen Jahren bereits mehrfach für Studien- und Projektarbeiten im Studiengang \enquote{Elektrotechnik – Automation} eingesetzt und kontinuierlich weiterentwickelt. Diese Arbeiten ermöglichten es den Studierenden, ein vertieftes Verständnis für Steuerungs- und Automatisierungssysteme zu erlangen und praxisorientierte Erfahrungen zu sammeln.
\subsection{Zielsetzung der Studienarbeit}
Trotz der bisherigen Entwicklungen besteht weiterhin Verbesserungspotenzial, insbesondere im Bereich der Sicherheit und der Programmstruktur.\\
Die vorliegende Studienarbeit befasst sich daher mit der Überarbeitung und Einführung neuer Sicherheitsmaßnahmen. Darüber hinaus soll eine auftragsbasierte Materialein- und -auslagerung implementiert werden. Weitere Schwerpunkte bilden die vollständige Implementierung der \textbf{RFID}-basierten Materialerkennung sowie die Umstellung des bestehenden SPS-Programms auf die Programmiersprache \textbf{Structured Text (ST)}.\\\\
Aufgrund des hohen Umfangs und der technischen Komplexität des Projekts erfolgt die vollständige Umsetzung in den \textbf{Theoriephasen 5 und 6}.\\
Ziel der aktuellen Bearbeitung (Theoriephase 5) ist die Einführung neuer Sicherheitskonzepte sowie die Überarbeitung der RFID-Materialerkennung und des SPS-Programms.\\\\
Abschließend soll die durchgeführte Arbeit die Grundlage für die vollständige Realisierung eines funktionsfähigen, sicheren und didaktisch wertvollen Hochregallagermodells bilden, das zukünftigen Studierenden als praxisnahes Lehr- und Forschungsobjekt dient.\\
Die in den folgenden Kapiteln erläuterten Schritte zur Planung und Umsetzung sollen verdeutlichen, wie die Studienarbeit (Teil 1) zur Weiterentwicklung der Funktionalität und der Sicherheit des Systems beiträgt. Nun erfolgt eine Erläuterung der Problemstellung und des aktuellen Standes der Technik.