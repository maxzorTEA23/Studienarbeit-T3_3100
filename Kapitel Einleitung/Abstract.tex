\section*{Zusammenfassung}
Diese Studienarbeit befasst sich mit der Weiterentwicklung eines automatisierten Modell‑Hochregallagers an der Dualen Hochschule Baden‑Württemberg. Ziel ist es, bestehende Funktionen um sichere und strukturierte Erweiterungen zu ergänzen und das System stärker an industrielle Anforderungen anzupassen. Der Fokus liegt auf der Integration einer sicherheitsgerichteten Lichtschranke, der Umsetzung eines internen Umlagerungsprozesses sowie der Optimierung der SPS‑Programmstruktur durch den Einsatz der Programmiersprache Structured Text (ST).
Die Lichtschranke wird über IO‑Link an eine Siemens S7‑1500 angebunden und mithilfe geeigneter Prozesswerte in die Sicherheitslogik integriert. Durch eine gezielte Auswertung des Prozesswertes „Number of Beams Occupied“ kann der Gefahrenbereich zuverlässig überwacht und bei unzulässigen Eingriffen eine automatische Abschaltung ausgelöst werden. Die Sicherheitsfunktion ist vollständig in die bestehende Steuerungsarchitektur eingebettet und mit dem Störungsmanagement gekoppelt.
Ergänzend wird ein Umlagerungsprozess realisiert, der die interne Verlagerung von Lagergütern ermöglicht. Eine eigenständige Schrittkette stellt einen sicheren und flexiblen Ablauf des Prozesses sicher und berücksichtigt sowohl die Bewegungslogik der Achsen als auch das Lagerplatzmanagement. Insgesamt wird das Hochregallagersystem funktional, sicherheitstechnisch und strukturell erweitert und bildet eine belastbare Grundlage für zukünftige, auftragsbasierte Erweiterungen.
\clearpage
\section*{Abstract}
This study focuses on the further development of an automated model high‑bay warehouse at the Baden‑Wuerttemberg Cooperative State University. The objective is to enhance existing functionalities by implementing safety‑related and structural improvements and to align the system more closely with industrial automation requirements. Key aspects of the work include the integration of a safety light curtain, the implementation of an internal relocation process, and the optimization of the PLC program structure using the programming language Structured Text (ST).
The safety light curtain is connected to a Siemens S7‑1500 PLC via IO‑Link and integrated into the control system using selected process values. By evaluating the “Number of Beams Occupied” parameter, the hazardous area can be reliably monitored and hazardous movements can be automatically stopped in the event of unauthorized access. The safety logic is fully embedded into the existing control architecture and linked to the fault management system.
In addition, a relocation process is implemented to enable the internal transfer of load units between different storage locations. A separate sequential control structure ensures a safe and transparent process flow while taking into account axis movements and storage location management. Overall, the high‑bay warehouse system is significantly improved in terms of functionality, safety, and software structure, providing a robust foundation for future extensions such as order‑based material flow control.