\clearpage
\subsection{Ausblick}
Die im Rahmen dieser Studienarbeit umgesetzten Erweiterungen bilden eine stabile technische und konzeptionelle Grundlage für die weitere Entwicklung des automatisierten Modell-Hochregallagers. Insbesondere durch die Integration der sicherheitsgerichteten Lichtschranke, die Einführung des Umlagerungsprozesses sowie die zunehmende Nutzung der Programmiersprache Structured Text wurde das System sowohl funktional als auch strukturell deutlich verbessert. Aufbauend auf diesen Ergebnissen wird sich die darauffolgende Arbeit auf die Implementierung eines auftragsbasierten Materialflusses konzentrieren.

Der zentrale Fokus der nächsten Entwicklungsstufe liegt auf der Einführung eines Auftragsmanagements mithilfe eines QR-Code-Readers. Ziel ist es, Ein- und Auslagerprozesse nicht mehr ausschließlich manuell oder fest vorgegeben auszulösen, sondern diese auf Basis eindeutig definierter Aufträge automatisch zu steuern. Hierdurch soll das Hochregallager stärker an reale industrielle Anwendungen angelehnt werden, in denen Materialbewegungen in der Regel durch digitale Aufträge aus übergeordneten Systemen angestoßen werden.

Im Rahmen der nächsten Arbeit soll ein QR-Reader als zusätzliche Identifikations- und Eingabeschnittstelle in das bestehende System integriert werden. Über den QR-Code sollen auftragsrelevante Informationen wie Lagerplatz, Material-ID, Prozessart (Einlagern, Auslagern oder Umlagern) sowie gegebenenfalls Prioritäten oder Zielpositionen erfasst werden. Die ausgelesenen Daten werden anschließend von der SPS verarbeitet und in strukturierter Form an die Prozesssteuerung übergeben. Damit entsteht eine direkte Verbindung zwischen physischer Auftragserfassung und automatisierter Anlagensteuerung.

Ein wesentlicher Bestandteil des Auftragsmanagements wird die Definition geeigneter Datenstrukturen innerhalb der SPS sein. Hierzu zählen unter anderem Auftragsobjekte, Warteschlangen sowie Statusinformationen zur Verfolgung des aktuellen Bearbeitungsstands. Durch die Verwendung strukturierter Datentypen in Structured Text können Aufträge klar definiert, verwaltet und sequenziell abgearbeitet werden. Dies ermöglicht eine saubere Trennung zwischen Auftragsebene und Bewegungs- bzw. Sicherheitslogik und erhöht die Modularität des Gesamtsystems.

Darüber hinaus bietet ein auftragsbasiertes System die Möglichkeit, bestehende Prozesse wie Einlagerung, Auslagerung und Umlagerung in einer übergeordneten Logik zusammenzuführen. Anstatt einzelne Funktionen separat auszulösen, kann der Anlagenbetrieb künftig vollständig über Aufträge gesteuert werden. Dies eröffnet Potenzial für eine automatische Priorisierung von Aufträgen, eine flexible Reihenfolge der Abarbeitung sowie eine bessere Auslastung der Anlage. Insbesondere in Kombination mit dem bereits implementierten Umlagerungsprozess lassen sich so optimierte Lagerstrategien realisieren.

Ein weiterer Aspekt der kommenden Arbeit betrifft die Erweiterung der Bedien- und Visualisierungsebene. Die über den QR-Reader eingelesenen Aufträge sollen auf dem HMI übersichtlich dargestellt werden, inklusive Informationen zum aktuellen Auftragsstatus, zur Restbearbeitungszeit und zu möglichen Störungen. Dadurch wird die Transparenz für den Bediener erhöht und der Anlagenzustand jederzeit nachvollziehbar dargestellt. Ergänzend könnte eine Historie abgeschlossener Aufträge implementiert werden, um Prozessabläufe analysieren und optimieren zu können.

Auch im Hinblick auf die Sicherheit ergeben sich neue Anforderungen. Das Auftragsmanagement muss eng mit den bestehenden Sicherheitsfunktionen verknüpft werden, sodass Aufträge bei aktiven Störungen oder unterbrochenen Schutzfunktionen nicht gestartet oder automatisch pausiert werden. Dadurch wird sichergestellt, dass der auftragsbasierte Betrieb jederzeit den sicherheitstechnischen Anforderungen entspricht und keine gefährlichen Zustände entstehen.

Zusammenfassend stellt die Implementierung eines QR-basierten Auftragsmanagements einen konsequenten nächsten Schritt in der Weiterentwicklung des automatisierten Modell-Hochregallagers dar. Sie ermöglicht eine realitätsnahe Abbildung moderner intralogistischer Prozesse und schafft die Grundlage für eine flexible, skalierbare und industrieorientierte Anlagensteuerung. Die Ergebnisse der vorliegenden Arbeit bilden hierfür eine solide Basis und ermöglichen es, das System in der folgenden Theoriephase gezielt um eine zentrale Funktion moderner Lagerlogistik zu erweitern.