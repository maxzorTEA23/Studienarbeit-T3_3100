\subsection{Prozessablauf der Umlagerung}
%Die \textbf{Beschreibung des Funktionsablaufs} stellt einen zentralen Bestandteil der technischen Dokumentation dar. Sie dient dazu, die einzelnen Schritte eines Prozesses in ihrer \textbf{logischen Reihenfolge} darzustellen und die zugrunde liegenden Bedingungen sowie Übergänge zwischen den Zuständen nachvollziehbar zu machen. Ein strukturierter Ablauf ist nicht nur für die Implementierung relevant, sondern bildet auch die Grundlage für die Analyse von \textbf{Prozesssicherheit}, \textbf{Effizienz} und \textbf{Fehleranfälligkeit}.

In diesem Kapitel wird der Funktionsablauf aus einer \textbf{systemorientierten Perspektive} betrachtet. Ziel ist es, die \textbf{Interaktion der beteiligten Komponenten} sowie die \textbf{Abhängigkeiten zwischen mechanischen Bewegungen, Steuerungslogik und Sicherheitsmechanismen} darzustellen. Darüber hinaus werden die wesentlichen Prozessschritte erläutert, die für einen \textbf{störungsfreien Betrieb} erforderlich sind. Die Darstellung erfolgt unabhängig von der konkreten Programmierung, um eine klare Trennung zwischen \textbf{konzeptioneller Beschreibung} und \textbf{technischer Umsetzung} zu gewährleisten.