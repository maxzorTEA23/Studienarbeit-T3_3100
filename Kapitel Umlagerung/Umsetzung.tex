\subsection{Umsetzung im TIA Portal}
Im Rahmen der Automatisierung der Modellhochregalanlage wurde eine Umlagerfunktion implementiert, die den Transport von Ladeeinheiten zwischen unterschiedlichen Lagerplätzen ermöglicht. Zur Realisierung dieser Funktion wurde eine Schrittkette eingesetzt, da sie eine strukturierte und übersichtliche Abbildung sequenzieller Abläufe erlaubt. Die Schrittkette dient hierbei als Steuerungslogik, um die einzelnen Prozessschritte – vom Anfahren des Quellplatzes über die Aufnahme der Ladeeinheit bis hin zur Ablage am Zielplatz – in einer definierten Reihenfolge auszuführen. Durch diese Vorgehensweise wird eine hohe Prozesssicherheit gewährleistet und die Komplexität der Steuerung reduziert.
