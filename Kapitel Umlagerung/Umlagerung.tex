% Kapitel Umlagerung


\section{Implementierung eines Umlagerungsprozesses}
\subsection{Definition \glqq Umlagerung\grqq{} im Kontext der Intralogistik}

In der Intralogistik bezeichnet eine Umlagerung die \textbf{interne Verlagerung von Material oder Ladeeinheiten innerhalb eines Lagersystems}, ohne dass ein Warenein- oder -ausgang erfolgt. Sie dient nicht der Erfüllung eines Kundenauftrags, sondern der \textbf{Optimierung der Lagerstruktur und der Betriebsabläufe}. Typische Gründe für Umlagerungen sind:

\begin{itemize}
	\item \textbf{Platzoptimierung}: Zusammenführen von Teilpaletten oder das Freiräumen bestimmter Stellplätze für neue Einlagerungen.
	\item \textbf{Reduktion der Zykluszeiten}: Lagerware, die weit weg von der Entnahmestelle eingelagert wurde, wird in Totzeiten des Systems näher an die Entnahmestelle umgelagert. Dies verkürzt die Zykluszeit bei späterer Anfrage der Ware.
	\item \textbf{Strategische Positionierung}: Umlagerung von Artikeln in Bereiche mit höherer Zugriffshäufigkeit, um Kommissionierzeiten zu reduzieren.
	\item \textbf{Qualitätssicherung}: Verbringen von Waren in Quarantäne- oder Prüfzonen.
	\item \textbf{Technische Anforderungen}: Freihalten von Stellplätzen für Wartungsarbeiten oder zur Vermeidung von Blockaden.
	\item \textbf{Bestandskorrekturen}: Anpassungen nach Inventuren oder bei fehlerhaften Buchungen.
\end{itemize}

Im Gegensatz zu \textbf{Ein- und Auslagerungen}, die externe Materialflüsse abbilden, ist die Umlagerung ein \textbf{rein interner Prozess}.

%Aufruf der Unterkapitel
\subsection{Prozessablauf der Umlagerung}
%Die \textbf{Beschreibung des Funktionsablaufs} stellt einen zentralen Bestandteil der technischen Dokumentation dar. Sie dient dazu, die einzelnen Schritte eines Prozesses in ihrer \textbf{logischen Reihenfolge} darzustellen und die zugrunde liegenden Bedingungen sowie Übergänge zwischen den Zuständen nachvollziehbar zu machen. Ein strukturierter Ablauf ist nicht nur für die Implementierung relevant, sondern bildet auch die Grundlage für die Analyse von \textbf{Prozesssicherheit}, \textbf{Effizienz} und \textbf{Fehleranfälligkeit}.

In diesem Kapitel wird der Funktionsablauf aus einer \textbf{systemorientierten Perspektive} betrachtet. Ziel ist es, die \textbf{Interaktion der beteiligten Komponenten} sowie die \textbf{Abhängigkeiten zwischen mechanischen Bewegungen, Steuerungslogik und Sicherheitsmechanismen} darzustellen. Darüber hinaus werden die wesentlichen Prozessschritte erläutert, die für einen \textbf{störungsfreien Betrieb} erforderlich sind. Die Darstellung erfolgt unabhängig von der konkreten Programmierung, um eine klare Trennung zwischen \textbf{konzeptioneller Beschreibung} und \textbf{technischer Umsetzung} zu gewährleisten.
\subsection{Umsetzung im TIA Portal}
Im Rahmen der Automatisierung der Modellhochregalanlage wurde eine Umlagerfunktion implementiert, die den Transport von Ladeeinheiten zwischen unterschiedlichen Lagerplätzen ermöglicht. Zur Realisierung dieser Funktion wurde eine Schrittkette eingesetzt, da sie eine strukturierte und übersichtliche Abbildung sequenzieller Abläufe erlaubt. Die Schrittkette dient hierbei als Steuerungslogik, um die einzelnen Prozessschritte – vom Anfahren des Quellplatzes über die Aufnahme der Ladeeinheit bis hin zur Ablage am Zielplatz – in einer definierten Reihenfolge auszuführen. Durch diese Vorgehensweise wird eine hohe Prozesssicherheit gewährleistet und die Komplexität der Steuerung reduziert.


\subsection{Visualisierung auf dem HMI}
Die Visualisierung des Umlagerungsprozesses wurde im bestehenden Stil des bisherigen HMI-Layouts umgesetzt, um eine konsistente Benutzerführung und ein einheitliches Erscheinungsbild zu gewährleisten. Sämtliche grafischen Elemente, Farbschemata und Bedienelemente orientieren sich an der zuvor implementierten Struktur für Ein- und Auslagerungsprozesse. Ergänzend wurde das HMI um spezifische Funktionen für die Umlagerung erweitert. Dazu zählen die Anzeige des aktuellen Quell- und Zielplatzes, die Positionsmeldungen während der Verfahrbewegungen sowie die Möglichkeit zur direkten Auswahl der Umlagerungsposition im Menü Übersicht.
\begin{figure}[H]
	\centering
	\includegraphics[width=\textwidth]{Screenshots/Umlagerung/HMI/Übersicht.png}
	\caption{Übersichtsdarstellung zur Auswahl des Umlagerplatzes im HMI}
	\label{fig:hmi_uebersicht}
\end{figure}

Oben abgebildet ist die Übersicht des Hochregallagers mit den einzelnen Lagerplätzen, die in einer vertikalen Anordnung dargestellt sind. Die Platznummern (z.\,B. 11, 21, 31, 41) setzen sich aus den X- und Y-Koordinaten zusammen , um eine schnelle Orientierung zu gewährleisten. Im Zentrum befindet sich ein Popup-Bereich für die Auswahl des Umlagerplatzes. Das Pop-Up wird geöffnet, sobald der Bediener der Anlage einen Lagerplatz anklickt und ihn  dadurch auswählt.

\begin{figure}[H]
	\centering
	\includegraphics[width=\textwidth]{Screenshots/Umlagerung/HMI/POP_UP.png}
	\caption{Popup-Fenster zur Anzeige von Produktinformationen und Eingabe des neuen Lagerplatzes}
	\label{fig:hmi_popup}
\end{figure}

Das Bild zeigt das Popup-Fenster, das zur Anzeige detaillierter Produktinformationen sowie zum Start des neuen Umlagerprozesses dient. Auf der linken Seite werden Statusinformationen des ausgewählten Lagerplatzes dargestellt, darunter der aktuelle Status, die Artikelnummer, das Material, die eindeutige Identifikationsnummer (UID) sowie ein Zeitstempel. Die rechte Seite des Fensters zeigt den zuvor in der Übersicht ausgewählten Koordinaten für die neue Zielposition (X- und Y-Koordinaten) sowie eine Schaltfläche zur Auslösung des Umlagerungsprozesses. Diese Struktur ermöglicht eine klare Trennung zwischen Informationsanzeige und Steuerungsfunktion.



\begin{figure}[H]
	\centering
	\includegraphics[width=\textwidth]{Screenshots/Umlagerung/HMI/Umalgerungsprozess.png}
	\caption{Manuelles Menü des Umlagerungsprozesses mit Anzeige der aktuellen Position}
	\label{fig:hmi_umlagerung}
\end{figure}

Abgebildet ist die Hauptansicht für die Durchführung der Umlagerung. Hier werden die Quell- und Zielpositionen in separaten Eingabebereichen dargestellt: Der obere Bereich (\textit{Von}) dient zur Eingabe der aktuellen Position (X-Soll, Y-Soll), während der untere Bereich (\textit{Nach}) die Zielposition aufnimmt. Darunter befindet sich eine Schaltfläche zur Auslösung des Umlagerungsprozesses. Rechts ist eine schematische Darstellung der Anlage integriert, die den räumlichen Bezug zum Hochregallager verdeutlicht. Diese Ansicht ermöglicht eine intuitive Bedienung und stellt alle relevanten Parameter für die Umlagerung kompakt dar.

\clearpage
\section{Fazit und Ausblick}
\subsection{Fazit}
Im Rahmen dieser Studienarbeit wurde das automatisierte Modell-Hochregallager der Dualen Hochschule Baden-Württemberg gezielt weiterentwickelt und um wesentliche funktionale sowie sicherheitstechnische Aspekte ergänzt. Ziel war es, bestehende Defizite im Bereich der Sicherheit, der Programmstruktur und der Prozesslogik zu analysieren und durch geeignete technische Maßnahmen zu beheben. Die durchgeführten Arbeiten leisten einen wichtigen Beitrag zur funktionalen Erweiterung und zur praxisnahen Ausgestaltung des Systems.

Ein zentraler Schwerpunkt der Arbeit lag auf der Integration einer berührungslos wirkenden Schutzeinrichtung in Form eines Lichtgatters. Durch die Implementierung der Lichtschranke in Verbindung mit einem IO-Link-fähigen Sensor konnte eine zuverlässige Überwachung des Gefahrenbereichs realisiert werden. Die detaillierte Analyse der verfügbaren Prozesswerte sowie die gezielte Auswahl des Modus „Number of Beams Occupied (NBO)“ ermöglichen eine klare Unterscheidung zwischen zulässigen Prozessbewegungen, wie dem Einfahren des Zylinders, und unzulässigen Eingriffen in den Arbeitsbereich. Dadurch wird eine situationsabhängige und zugleich robuste Sicherheitsabschaltung gewährleistet.

Die sicherheitstechnische Logik wurde vollständig in die bestehende SPS-Architektur integriert und über einen eigenen Funktionsbaustein realisiert. Durch die zentrale Freigabevariable wird sichergestellt, dass sämtliche Aktoren des Systems nur dann angesteuert werden, wenn keine Störung vorliegt und der Gefahrenbereich frei ist. Dieses Vorgehen erhöht nicht nur die Sicherheit des Systems, sondern verbessert auch die Nachvollziehbarkeit und Wartbarkeit der Steuerungssoftware. Die Kopplung der Sicherheitsfunktion mit dem vorhandenen Störungsmanagement stellt sicher, dass sicherheitsrelevante Zustände eindeutig erkannt, verarbeitet und visualisiert werden.

Ein weiterer wesentlicher Bestandteil der Arbeit war die Einführung eines Umlagerungsprozesses. Im Gegensatz zu den bereits vorhandenen Ein- und Auslagerungsfunktionen handelt es sich hierbei um einen rein internen logistischen Prozess, der zur Optimierung der Lagerstruktur dient. Die Entwicklung einer eigenständigen Schrittkette erwies sich als notwendig, um die zusätzlichen Anforderungen hinsichtlich Bewegungslogik, Verzweigungen und Prozesssicherheit abzubilden. Durch die Kombination bestehender Grundlogiken mit neuen Entscheidungsstrukturen konnte ein flexibler und nachvollziehbarer Umlagerungsablauf realisiert werden.

Darüber hinaus zeigt die Arbeit deutlich die Vorteile der textbasierten Programmierung in Structured Text (ST) gegenüber grafischen Programmiersprachen wie FUP. Insbesondere bei komplexeren Abläufen, sicherheitsrelevanten Abfragen und mathematischen Vergleichen bietet ST eine deutlich höhere Übersichtlichkeit und Skalierbarkeit. Die Umstellung einzelner Funktionsbereiche auf ST trägt langfristig zu einer besseren Lesbarkeit, einfacheren Fehlersuche und verbesserten Erweiterbarkeit des Programms bei.

Zusammenfassend lässt sich festhalten, dass die in dieser Studienarbeit umgesetzten Maßnahmen die funktionale Sicherheit, Prozessstabilität und didaktische Qualität des Modell-Hochregallagers deutlich verbessern. Das System bildet nun nicht nur grundlegende logistische Prozesse ab, sondern berücksichtigt auch sicherheitstechnische Anforderungen, wie sie in realen industriellen Anlagen zwingend erforderlich sind. Damit stellt das Hochregallager eine praxisnahe und zukunftsfähige Lernplattform für Studierende im Bereich der Automatisierungstechnik dar.
\clearpage
\subsection{Ausblick}
Die im Rahmen dieser Studienarbeit umgesetzten Erweiterungen bilden eine stabile technische und konzeptionelle Grundlage für die weitere Entwicklung des automatisierten Modell-Hochregallagers. Insbesondere durch die Integration der sicherheitsgerichteten Lichtschranke, die Einführung des Umlagerungsprozesses sowie die zunehmende Nutzung der Programmiersprache Structured Text wurde das System sowohl funktional als auch strukturell deutlich verbessert. Aufbauend auf diesen Ergebnissen wird sich die darauffolgende Arbeit auf die Implementierung eines auftragsbasierten Materialflusses konzentrieren.

Der zentrale Fokus der nächsten Entwicklungsstufe liegt auf der Einführung eines Auftragsmanagements mithilfe eines QR-Code-Readers. Ziel ist es, Ein- und Auslagerprozesse nicht mehr ausschließlich manuell oder fest vorgegeben auszulösen, sondern diese auf Basis eindeutig definierter Aufträge automatisch zu steuern. Hierdurch soll das Hochregallager stärker an reale industrielle Anwendungen angelehnt werden, in denen Materialbewegungen in der Regel durch digitale Aufträge aus übergeordneten Systemen angestoßen werden.

Im Rahmen der nächsten Arbeit soll ein QR-Reader als zusätzliche Identifikations- und Eingabeschnittstelle in das bestehende System integriert werden. Über den QR-Code sollen auftragsrelevante Informationen wie Lagerplatz, Material-ID, Prozessart (Einlagern, Auslagern oder Umlagern) sowie gegebenenfalls Prioritäten oder Zielpositionen erfasst werden. Die ausgelesenen Daten werden anschließend von der SPS verarbeitet und in strukturierter Form an die Prozesssteuerung übergeben. Damit entsteht eine direkte Verbindung zwischen physischer Auftragserfassung und automatisierter Anlagensteuerung.

Ein wesentlicher Bestandteil des Auftragsmanagements wird die Definition geeigneter Datenstrukturen innerhalb der SPS sein. Hierzu zählen unter anderem Auftragsobjekte, Warteschlangen sowie Statusinformationen zur Verfolgung des aktuellen Bearbeitungsstands. Durch die Verwendung strukturierter Datentypen in Structured Text können Aufträge klar definiert, verwaltet und sequenziell abgearbeitet werden. Dies ermöglicht eine saubere Trennung zwischen Auftragsebene und Bewegungs- bzw. Sicherheitslogik und erhöht die Modularität des Gesamtsystems.

Darüber hinaus bietet ein auftragsbasiertes System die Möglichkeit, bestehende Prozesse wie Einlagerung, Auslagerung und Umlagerung in einer übergeordneten Logik zusammenzuführen. Anstatt einzelne Funktionen separat auszulösen, kann der Anlagenbetrieb künftig vollständig über Aufträge gesteuert werden. Dies eröffnet Potenzial für eine automatische Priorisierung von Aufträgen, eine flexible Reihenfolge der Abarbeitung sowie eine bessere Auslastung der Anlage. Insbesondere in Kombination mit dem bereits implementierten Umlagerungsprozess lassen sich so optimierte Lagerstrategien realisieren.

Ein weiterer Aspekt der kommenden Arbeit betrifft die Erweiterung der Bedien- und Visualisierungsebene. Die über den QR-Reader eingelesenen Aufträge sollen auf dem HMI übersichtlich dargestellt werden, inklusive Informationen zum aktuellen Auftragsstatus, zur Restbearbeitungszeit und zu möglichen Störungen. Dadurch wird die Transparenz für den Bediener erhöht und der Anlagenzustand jederzeit nachvollziehbar dargestellt. Ergänzend könnte eine Historie abgeschlossener Aufträge implementiert werden, um Prozessabläufe analysieren und optimieren zu können.

Auch im Hinblick auf die Sicherheit ergeben sich neue Anforderungen. Das Auftragsmanagement muss eng mit den bestehenden Sicherheitsfunktionen verknüpft werden, sodass Aufträge bei aktiven Störungen oder unterbrochenen Schutzfunktionen nicht gestartet oder automatisch pausiert werden. Dadurch wird sichergestellt, dass der auftragsbasierte Betrieb jederzeit den sicherheitstechnischen Anforderungen entspricht und keine gefährlichen Zustände entstehen.

Zusammenfassend stellt die Implementierung eines QR-basierten Auftragsmanagements einen konsequenten nächsten Schritt in der Weiterentwicklung des automatisierten Modell-Hochregallagers dar. Sie ermöglicht eine realitätsnahe Abbildung moderner intralogistischer Prozesse und schafft die Grundlage für eine flexible, skalierbare und industrieorientierte Anlagensteuerung. Die Ergebnisse der vorliegenden Arbeit bilden hierfür eine solide Basis und ermöglichen es, das System in der folgenden Theoriephase gezielt um eine zentrale Funktion moderner Lagerlogistik zu erweitern.