% Kapitel Umlagerung


\section{Implementierung eines Umlagerungsprozesses}
\subsection{Definition \glqq Umlagerung\grqq{} im Kontext der Intralogistik}

In der \textbf{Intralogistik} bezeichnet eine \textbf{Umlagerung} die \textbf{interne Verlagerung von Material oder Ladeeinheiten innerhalb eines Lagersystems}, ohne dass ein Warenein- oder -ausgang erfolgt. Sie dient nicht der Erfüllung eines Kundenauftrags, sondern der \textbf{Optimierung der Lagerstruktur und der Betriebsabläufe}. Typische Gründe für Umlagerungen sind:

\begin{itemize}
	\item \textbf{Platzoptimierung}: Zusammenführen von Teilpaletten oder das Freiräumen bestimmter Stellplätze für neue Einlagerungen.
	\item \textbf{Strategische Positionierung}: Umlagerung von Artikeln in Bereiche mit höherer Zugriffshäufigkeit (ABC-Analyse), um Kommissionierzeiten zu reduzieren.
	\item \textbf{Qualitätssicherung}: Verbringen von Waren in Quarantäne- oder Prüfzonen.
	\item \textbf{Technische Anforderungen}: Freihalten von Stellplätzen für Wartungsarbeiten oder zur Vermeidung von Blockaden.
	\item \textbf{Bestandskorrekturen}: Anpassungen nach Inventuren oder bei fehlerhaften Buchungen.
\end{itemize}

Im Gegensatz zu \textbf{Ein- und Auslagerungen}, die externe Materialflüsse abbilden, ist die Umlagerung ein \textbf{rein interner Prozess}.

%Aufruf der Unterkapitel
\subsection{Prozessablauf der Umlagerung}
%Die \textbf{Beschreibung des Funktionsablaufs} stellt einen zentralen Bestandteil der technischen Dokumentation dar. Sie dient dazu, die einzelnen Schritte eines Prozesses in ihrer \textbf{logischen Reihenfolge} darzustellen und die zugrunde liegenden Bedingungen sowie Übergänge zwischen den Zuständen nachvollziehbar zu machen. Ein strukturierter Ablauf ist nicht nur für die Implementierung relevant, sondern bildet auch die Grundlage für die Analyse von \textbf{Prozesssicherheit}, \textbf{Effizienz} und \textbf{Fehleranfälligkeit}.

In diesem Kapitel wird der Funktionsablauf aus einer \textbf{systemorientierten Perspektive} betrachtet. Ziel ist es, die \textbf{Interaktion der beteiligten Komponenten} sowie die \textbf{Abhängigkeiten zwischen mechanischen Bewegungen, Steuerungslogik und Sicherheitsmechanismen} darzustellen. Darüber hinaus werden die wesentlichen Prozessschritte erläutert, die für einen \textbf{störungsfreien Betrieb} erforderlich sind. Die Darstellung erfolgt unabhängig von der konkreten Programmierung, um eine klare Trennung zwischen \textbf{konzeptioneller Beschreibung} und \textbf{technischer Umsetzung} zu gewährleisten.
\subsection{Umsetzung im TIA Portal}
Im Rahmen der Automatisierung der Modellhochregalanlage wurde eine Umlagerfunktion implementiert, die den Transport von Ladeeinheiten zwischen unterschiedlichen Lagerplätzen ermöglicht. Zur Realisierung dieser Funktion wurde eine Schrittkette eingesetzt, da sie eine strukturierte und übersichtliche Abbildung sequenzieller Abläufe erlaubt. Die Schrittkette dient hierbei als Steuerungslogik, um die einzelnen Prozessschritte – vom Anfahren des Quellplatzes über die Aufnahme der Ladeeinheit bis hin zur Ablage am Zielplatz – in einer definierten Reihenfolge auszuführen. Durch diese Vorgehensweise wird eine hohe Prozesssicherheit gewährleistet und die Komplexität der Steuerung reduziert.


\subsection{Visualisierung auf dem HMI}
Die Visualisierung des Umlagerungsprozesses wurde im bestehenden Stil des bisherigen HMI-Layouts umgesetzt, um eine konsistente Benutzerführung und ein einheitliches Erscheinungsbild zu gewährleisten. Sämtliche grafischen Elemente, Farbschemata und Bedienelemente orientieren sich an der zuvor implementierten Struktur für Ein- und Auslagerungsprozesse. Ergänzend wurde das HMI um spezifische Funktionen für die Umlagerung erweitert. Dazu zählen die Anzeige des aktuellen Quell- und Zielplatzes, die Positionsmeldungen während der Verfahrbewegungen sowie die Möglichkeit zur direkten Auswahl der Umlagerungsposition im Menü Übersicht.
\begin{figure}[H]
	\centering
	\includegraphics[width=\textwidth]{Screenshots/Umlagerung/HMI/Übersicht.png}
	\caption{Übersichtsdarstellung zur Auswahl des Umlagerplatzes im HMI}
	\label{fig:hmi_uebersicht}
\end{figure}

Oben abgebildet ist die Übersicht des Hochregallagers mit den einzelnen Lagerplätzen, die in einer vertikalen Anordnung dargestellt sind. Die Platznummern (z.\,B. 11, 21, 31, 41) setzen sich aus den X- und Y-Koordinaten zusammen , um eine schnelle Orientierung zu gewährleisten. Im Zentrum befindet sich ein Popup-Bereich für die Auswahl des Umlagerplatzes. Das Pop-Up wird geöffnet, sobald der Bediener der Anlage einen Lagerplatz anklickt und ihn  dadurch auswählt.

\begin{figure}[H]
	\centering
	\includegraphics[width=\textwidth]{Screenshots/Umlagerung/HMI/POP_UP.png}
	\caption{Popup-Fenster zur Anzeige von Produktinformationen und Eingabe des neuen Lagerplatzes}
	\label{fig:hmi_popup}
\end{figure}

Das Bild zeigt das Popup-Fenster, das zur Anzeige detaillierter Produktinformationen sowie zum Start des neuen Umlagerprozesses dient. Auf der linken Seite werden Statusinformationen des ausgewählten Lagerplatzes dargestellt, darunter der aktuelle Status, die Artikelnummer, das Material, die eindeutige Identifikationsnummer (UID) sowie ein Zeitstempel. Die rechte Seite des Fensters zeigt den zuvor in der Übersicht ausgewählten Koordinaten für die neue Zielposition (X- und Y-Koordinaten) sowie eine Schaltfläche zur Auslösung des Umlagerungsprozesses. Diese Struktur ermöglicht eine klare Trennung zwischen Informationsanzeige und Steuerungsfunktion.



\begin{figure}[H]
	\centering
	\includegraphics[width=\textwidth]{Screenshots/Umlagerung/HMI/Umalgerungsprozess.png}
	\caption{Manuelles Menü des Umlagerungsprozesses mit Anzeige der aktuellen Position}
	\label{fig:hmi_umlagerung}
\end{figure}

Abgebildet ist die Hauptansicht für die Durchführung der Umlagerung. Hier werden die Quell- und Zielpositionen in separaten Eingabebereichen dargestellt: Der obere Bereich (\textit{Von}) dient zur Eingabe der aktuellen Position (X-Soll, Y-Soll), während der untere Bereich (\textit{Nach}) die Zielposition aufnimmt. Darunter befindet sich eine Schaltfläche zur Auslösung des Umlagerungsprozesses. Rechts ist eine schematische Darstellung der Anlage integriert, die den räumlichen Bezug zum Hochregallager verdeutlicht. Diese Ansicht ermöglicht eine intuitive Bedienung und stellt alle relevanten Parameter für die Umlagerung kompakt dar.