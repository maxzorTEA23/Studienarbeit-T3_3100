
\clearpage
\section{Fazit und Ausblick}
\subsection{Fazit}
Im Rahmen dieser Studienarbeit wurde das automatisierte Modell-Hochregallager der Dualen Hochschule Baden-Württemberg gezielt weiterentwickelt und um wesentliche funktionale sowie sicherheitstechnische Aspekte ergänzt. Ziel war es, bestehende Defizite im Bereich der Sicherheit, der Programmstruktur und der Prozesslogik zu analysieren und durch geeignete technische Maßnahmen zu beheben. Die durchgeführten Arbeiten leisten einen wichtigen Beitrag zur funktionalen Erweiterung und zur praxisnahen Ausgestaltung des Systems.

Ein zentraler Schwerpunkt der Arbeit lag auf der Integration einer berührungslos wirkenden Schutzeinrichtung in Form eines Lichtgatters. Durch die Implementierung der Lichtschranke in Verbindung mit einem IO-Link-fähigen Sensor konnte eine zuverlässige Überwachung des Gefahrenbereichs realisiert werden. Die detaillierte Analyse der verfügbaren Prozesswerte sowie die gezielte Auswahl des Modus „Number of Beams Occupied (NBO)“ ermöglichen eine klare Unterscheidung zwischen zulässigen Prozessbewegungen, wie dem Einfahren des Zylinders, und unzulässigen Eingriffen in den Arbeitsbereich. Dadurch wird eine situationsabhängige und zugleich robuste Sicherheitsabschaltung gewährleistet.

Die sicherheitstechnische Logik wurde vollständig in die bestehende SPS-Architektur integriert und über einen eigenen Funktionsbaustein realisiert. Durch die zentrale Freigabevariable wird sichergestellt, dass sämtliche Aktoren des Systems nur dann angesteuert werden, wenn keine Störung vorliegt und der Gefahrenbereich frei ist. Dieses Vorgehen erhöht nicht nur die Sicherheit des Systems, sondern verbessert auch die Nachvollziehbarkeit und Wartbarkeit der Steuerungssoftware. Die Kopplung der Sicherheitsfunktion mit dem vorhandenen Störungsmanagement stellt sicher, dass sicherheitsrelevante Zustände eindeutig erkannt, verarbeitet und visualisiert werden.

Ein weiterer wesentlicher Bestandteil der Arbeit war die Einführung eines Umlagerungsprozesses. Im Gegensatz zu den bereits vorhandenen Ein- und Auslagerungsfunktionen handelt es sich hierbei um einen rein internen logistischen Prozess, der zur Optimierung der Lagerstruktur dient. Die Entwicklung einer eigenständigen Schrittkette erwies sich als notwendig, um die zusätzlichen Anforderungen hinsichtlich Bewegungslogik, Verzweigungen und Prozesssicherheit abzubilden. Durch die Kombination bestehender Grundlogiken mit neuen Entscheidungsstrukturen konnte ein flexibler und nachvollziehbarer Umlagerungsablauf realisiert werden.

Darüber hinaus zeigt die Arbeit deutlich die Vorteile der textbasierten Programmierung in Structured Text (ST) gegenüber grafischen Programmiersprachen wie FUP. Insbesondere bei komplexeren Abläufen, sicherheitsrelevanten Abfragen und mathematischen Vergleichen bietet ST eine deutlich höhere Übersichtlichkeit und Skalierbarkeit. Die Umstellung einzelner Funktionsbereiche auf ST trägt langfristig zu einer besseren Lesbarkeit, einfacheren Fehlersuche und verbesserten Erweiterbarkeit des Programms bei.

Zusammenfassend lässt sich festhalten, dass die in dieser Studienarbeit umgesetzten Maßnahmen die funktionale Sicherheit, Prozessstabilität und didaktische Qualität des Modell-Hochregallagers deutlich verbessern. Das System bildet nun nicht nur grundlegende logistische Prozesse ab, sondern berücksichtigt auch sicherheitstechnische Anforderungen, wie sie in realen industriellen Anlagen zwingend erforderlich sind. Damit stellt das Hochregallager eine praxisnahe und zukunftsfähige Lernplattform für Studierende im Bereich der Automatisierungstechnik dar.