\subsubsection{Sicherheit in der Automatisierung}
Im Kontext der Automatisierung von technischen Anlagen ist die Sicherheit der größte und wichtigste Aspekt.\\
Beim Betreiben eines Systems muss vollständig ausgeschlossen sein, dass keine Gefahr für Menschen, von der Anlage, ausgeht. Das bedeutet, bei Planung der Anlage, steht die Sicherheit im Fokus.\\
Speziell in der Produktion bzw. in der Logistik sind sicherheitstechnische Aspekte enorm wichtig.\\
Bei automatisierten Prozessen wie z.B. das Verfahren eines Hochregallagers, können Gefahren durch Implementierung von Sicherheitslogiken ausgeschlossen werden.\\
Das Ziel ist immer Gefährdungen bzw. Verletzungen von Menschen zu verhindern.
\subsubsection*{Einordnung in den Kontext der funktionalen Sicherheit}
Um die Sicherheitsprinzipien besser zu verstehen ist es wichtig die einzelnen Aspekte der Sicherheitsthematik in den Gesamtkontext einzuordnen.\\
In diesem Zusammenhang spielt der Bereich funktionale Sicherheit eine entscheidende Rolle.\\
Hierunter versteht man den Teil einer Maschine der von den korrekt funktionierenden Steuerungs- und Sicherheitskomponenten abhängt. Sie sorgt dafür, das bei kritischen Situationen ein sicherer Zustand der Anlage erreicht wird. EIn Beispiel hierfür wäre das direkte Abschalten eines Motors.\\
Damit sicherheitsbezogene Steuerungssysteme zuverlässig arbeiten, müssen sie bestimmte Anforderungen erfüllen, die in internationalen Normen festgelegt sind.\\
Einige dieser Vorschriften sind im folgenden aufgelistet:
\begin{itemize}
	\item \textbf{EN ISO 13849-1}: Beschreibt die sicherheitsrelevanten Teile einer Anlage, u.A. \textbf{PL (Performance Level)}
	\item \textbf{IEC 62061}: Behandelt die funktionale Sicherheit elektrischer, elektronischer und programmierbarer Steuerungssysteme. Baisert auf \textbf{SIL} (Safety Integrity Level )
	\item
\end{itemize}