\subsubsection{Sicherheit in der Automatisierung}
Im Kontext der Automatisierung von technischen Anlagen ist die Sicherheit der größte und wichtigste Aspekt \cite{Bernstein.2016}.\\
Beim Betreiben eines Systems muss vollständig ausgeschlossen sein, dass keine Gefahr für Menschen, von der Anlage, ausgeht. Das bedeutet, bei Planung der Anlage, steht die Sicherheit im Fokus.\\
Speziell in der Produktion bzw. in der Logistik sind sicherheitstechnische Aspekte enorm wichtig.\\
Bei automatisierten Prozessen wie z.B. das Verfahren eines Hochregallagers, können Gefahren durch Implementierung von Sicherheitslogiken ausgeschlossen werden.\\
Das Ziel ist immer Gefährdungen bzw. Verletzungen von Menschen zu verhindern.

\subsubsection*{Einordnung in den Kontext der funktionalen Sicherheit}
Um die Sicherheitsprinzipien besser zu verstehen, ist es wichtig, die einzelnen Aspekte der Sicherheitsthematik im Gesamtkontext einzuordnen.\\
In diesem Zusammenhang spielt der Bereich der funktionalen Sicherheit eine entscheidende Rolle.\\
Hierunter versteht man den Teil einer Maschine, der von den korrekt funktionierenden Steuerungs- und Sicherheitskomponenten abhängt. Sie sorgt dafür, dass bei kritischen Situationen ein sicherer Zustand der Anlage erreicht wird. Ein Beispiel hierfür wäre das direkte Abschalten eines Motors\cite{funktionale_Sicherheit}.\\
Damit sicherheitsbezogene Steuerungssysteme zuverlässig arbeiten, müssen sie bestimmte Anforderungen erfüllen, die in internationalen Normen festgelegt sind.\\
Einige dieser Vorschriften sind im Folgenden aufgelistet\cite{funktionale_Sicherheit}:
\begin{itemize}
	\item \textbf{EN ISO 13849-1}: Beschreibt die sicherheitsrelevanten Teile einer Anlage, unter anderem den \textbf{PL (Performance Level)}.
	\item \textbf{IEC 61508}: Behandelt die funktionale Sicherheit elektrischer, elektronischer und programmierbarer Steuerungssysteme. Basiert auf dem \textbf{SIL (Safety Integrity Level)}.
	\item \textbf{EN 61496}: Sicherheit von Maschinen – Berührungslos wirkende Schutzeinrichtungen (BWS), beispielsweise Lichtschranken oder Lichtvorhänge.
\end{itemize}
Die genannten Normen zeigen ebenfalls die Bedeutung der Sicherheit im Anlagenbau für den Gesetzgeber. Falls es zu Verstößen kommen sollte, führt dies zu Konsequenzen für den Arbeitgeber.\\
Das ist der Grund dafür, dass Unternehmen sehr genau auf die Einhaltung dieser Vorschriften achten.\\
Für die genannten Normen gibt es dementsprechende Geräte, welche beispielsweise im Fehlerfall (Fehlerstrom) abschalten sollen (FI-Schutzschalter).\\
Die Auswahl der Sicherheitskomponenten erfolgt auf Basis einer sogenannten \textbf{Risikobeurteilung}. Diese ermittelt die potenziellen Gefahren und legt das \textbf{Performance Level} bzw. den \textbf{SIL} fest. In der Praxis werden Sicherheitsfunktionen häufig mit einem \textbf{PL~d} oder \textbf{SIL~2} realisiert, da diese ein gutes Verhältnis zwischen Aufwand und Sicherheit bieten.\\

Eine zentrale Rolle innerhalb dieser sicherheitsrelevanten Systeme spielt die \textbf{Sensorik}. Sie dient als Grundlage zur Erfassung von Zuständen und Bewegungen innerhalb einer Anlage und bildet somit die Basis für jede Sicherheitsfunktion.\\
Besonders in automatisierten Prozessen ist es wichtig, dass Gefahrenzonen zuverlässig überwacht werden, ohne den Produktionsablauf zu beeinträchtigen.\\
Hier kommen berührungslos wirkende Schutzeinrichtungen zum Einsatz, wie beispielsweise \textbf{Lichtschranken} oder \textbf{Lichtvorhänge}. Diese ermöglichen eine sichere und zugleich flexible Absicherung von Arbeitsbereichen, an denen Mensch und Maschine aufeinandertreffen.\\
Lichtschranken zählen zu den am weitesten verbreiteten Sicherheitseinrichtungen in der Industrie und werden unter anderem bei Robotersystemen, Förderanlagen oder automatisierten Handhabungsstationen verwendet.\\
Im folgenden Abschnitt wird die Funktionsweise, der Aufbau sowie der Einsatz von Lichtschranken in sicherheitsrelevanten Anwendungen näher erläutert.
\subsubsection{Lichtschranken in sicherheitsrelevanten Anlagen}
Die Lichtschranke ist eine der wichtigsten Sicherheitseinrichtungen in der Fertigung und Logistik.\\
Sie verhindert das unbefugte Eingreifen in laufende Prozesse, sodass mögliche Gefahren für Personen zuverlässig ausgeschlossen werden können.\\
Viele Hersteller bieten ihre Bauteile frei konfigurierbar an. Dadurch ist es möglich, die Betriebsmodi und Parameter individuell an die jeweilige Anwendung anzupassen, beispielsweise bei der Absicherung von Hochregallagern oder automatisierten Förderanlagen.\\
Lichtschranken kommen in sicherheitsrelevanten Anlagen überall dort zum Einsatz, wo Gefahrenbereiche zuverlässig überwacht werden müssen, ohne dass mechanische Barrieren den Prozessfluss behindern.
\subsubsection*{Funktionalität einer Lichtschranke}
Bei einer Lichtschranke handelt es sich um eine berührungslos wirkende Schutzeinrichtung, die mit einem oder mehreren Lichtstrahlen arbeitet. Diese Lichtstrahlen bilden eine optische Barriere, die bei Unterbrechung ein Signal auslöst.\\
Eine Lichtschranke besteht grundsätzlich aus zwei Modulen: einem \textbf{Sender} und einem \textbf{Empfänger}. Der Sender erzeugt ein Lichtsignal, das vom Empfänger erfasst wird. Wird der Lichtstrahl unterbrochen, erkennt die Auswerteelektronik diesen Zustand und leitet eine vordefinierte Sicherheitsreaktion ein – beispielsweise das Abschalten eines Antriebs oder das Anhalten eines Roboters.\\
Die am häufigsten verwendeten Lichtschrankentypen sind die \textbf{Reflexionslichtschranke} und die \textbf{Infrarotlichtschranke}\cite{Lichtschranke}.\\
Bei einer \textbf{Reflexionslichtschranke} befinden sich Sender und Empfänger im selben Gehäuse. Das ausgesendete Lichtsignal wird über einen Reflektor zurückgeworfen und von der Empfangseinheit detektiert. Wird der Lichtstrahl durch ein Objekt unterbrochen, bleibt das Signal aus und die Sicherheitsfunktion wird aktiviert.\\
Die im Projekt verwendete Variante ist eine \textbf{Infrarotlichtschranke}. Diese arbeitet mit unsichtbarem Infrarotlicht, das unempfindlich gegenüber Umgebungslicht und Staub ist.\\
\begin{figure}[H]
	\centering
	\includegraphics[width=0.6\linewidth]{images/Lichtschranke}
	\caption{Schematische Darstellung des IFM Lichtgatters\cite{Lichtgatter_ifm_2}}
	\label{fig: Lichtgatter}
\end{figure}
Hierbei wird ein kontinuierlicher Infrarotstrahl vom Sender ausgesendet, der vom Empfänger registriert wird. Wird der Strahl unterbrochen, erkennt die Steuerung dies sofort als Störung oder Eingriff und löst eine Sicherheitsreaktion aus \ref{fig: Lichtgatter}.\\
Durch die hohe Reichweite und Störsicherheit eignet sich die Infrarotlichtschranke besonders für industrielle Umgebungen, in denen eine präzise und zuverlässige Objekterkennung gefordert ist.\\
Häufig variiert die Bezeichnung solcher Systeme. Viele Hersteller verwenden dafür auch den Begriff \textbf{Lichtgitter} oder \textbf{Lichtvorhang}.\\
So auch der Hersteller \textbf{ifm}, der das im Projekt verwendete Produkt bereitstellt\cite{Lichtgatter_ifm}. Diese Systeme sind modular aufgebaut und lassen sich über verschiedene Schnittstellen – beispielsweise IO-Link – direkt in eine Steuerung integrieren.